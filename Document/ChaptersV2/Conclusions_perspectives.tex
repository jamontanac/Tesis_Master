\chapter{Conclusions and Perspectives}

Throughout all this document, we have exposed a series of arguments to show, the way Ultra Orthogonality could be an alternative mechanism for understanding equilibration.  As we mentioned, this is connected to thermalisation problem, since for most systems thermal typicality is expected, thus the state of equilibrium we end with has to be nothing but the canonical state. Additionally, we tackle the problem when Ultra Orthogonality holds exactly, that means, the equilibrium state is automatically reached by computing the correspondent reduced state of the Universe. By this, we mean that we have led all our efforts to show that for the case of Fermionic systems, there exists a particular sort of pure dynamical states, such that when taking its correspondent partial trace over the environment, these resultant reduced states are automatically in its correspondent equilibrium, and the way to understand this phenomena, can be through an information theory viewpoint, whereby, the fact that we work over a finite algebra, as the one in fermions, let us formulate this particular problem in terms of minimum distance codes.\\



\indent Not satisfied with finding this result in a virtual general way for Fermionic systems, we decided to estimate the size of the correspondent Hilbert subspace associated with the states that fulfil Ultra Orthogonality, in the sense that all reduced states, will be automatically stationary states (Constant states over time). Thanks the formalism of exponent errors, used in Code Theory , it was possible to provide an expression to the large deviation present in the Fermionic systems, namely, this exponent tell us the an estimate about the expected number of random Fermionic minimum codes that fulfil Ultra Orthogonality, Which as we show, whenever this exponent is larger than zero, we therefore are able to show that the correspondent Hilbert subspace to these states is exponentially large. In other words, aside of our result being quite general, we found that whenever the correspondent exponent, which depends on the parameters of the system, is larger than zero, we will have a huge Hilbert subspace (exponentially large), which contains states such that when taking the partial trace over its correspondent environment are automatically equilibrated.\\

\indent This particular result bring more questions than answers. One of the first questions we may ask is, what happens if instead of thinking this problem only in terms of random codes, we impose a condition over the energy?. From our point of view, we consider that this could be the main key to connect all this results with the thermalisation problem, indeed we think that even though, the the exponent error must change, we could still find situations in which it is grater than zero, and therefore, find that the expected number of codes at certain energy will also follow a large deviation law. The problem we stumble upon was that when imposing a restriction over the energy for the codewords, liner superposition may end up not conserving the restriction of energy, so another study will be needed to confirm if this can be done and provide the connection to thermalisation.

\indent Finally, we want to point out that the physical meaning of these quantities was not discussed, and maybe the question of  what exactly is the physical meaning of these exponents? could have emerge at some point without any answer. What we wanted to stress here is that indeed, there was an attempt to find some physical interpretation about these quantities. Specifically, we tried to tackled this matter by studying the one dimensional $XY$ model of spins $1/2$. The choice for this particular system, was due to the fact that this particular system can be analytically solved with efficient algorithms. In spite of our efforts, hitherto, a complete interpretation or an intuition of our result have not been found. Thus we emphasise this point as a future work to be done in this subject.




