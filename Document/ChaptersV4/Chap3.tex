\chapter{Studying ultra-orthogonality in fermionic systems}
\noindent In the last two chapters we have discussed the foundations of statistical mechanics and we provided the needed theoretical background to delimit our problem to the case of a special kind of fermionic systems. Specifically, in the first chapter we discussed the foundations of statistical mechanics and we introduced typicality as an individualist approach of thermalisation, then we motivated the property of ultra-orthogonality as a consequence of typicality and we showed that ultra-orthogonality is related to equilibration as an instantaneous phenomenon. Afterwards, in the second chapter, we introduced all the theoretical background to delimit our problem to a special kind of fermionic systems, finally in the last section, we talked about coding theory and we showed an important result in random codes. This chapter will be devoted to connect all the concepts presented in previous chapters and show an explicit connection between random code theory and fermionic systems. Moreover, we will show how this connection allows us to show that for a specific kind of fermionic systems, it is possible to find exponentially large Hilbert subspaces such that ultra-orthogonality can be guaranteed.
\section{Delimiting our problem}
In order to delimit our problem we have first to recall some concepts that were introduced previously. First, in the first part of the second chapter, we introduced the fermionic quadratic Hamiltonians in the operators of creation/annihilation. As we mentioned in that moment, these quadratic Hamiltonians have the interesting property that not only the ground state but every eigenstate representing a certain number of 
excitations of quasi-particles, belong to the so-called class of fermionic Gaussian states. We also discussed that this particular class of states are completely characterised by its second order correlations as the Wick theorem states. This characterisation motivated us to study these kind of systems in terms of the Majorana fermions $\{\hat{\gamma}_i\}$. The algebra generated by the operators $\{\hat{\gamma}_i\}$ is known as the Clifford algebra, which for the case of systems of $N$ modes we denoted by $\mathcal{C}_{2N}$. Since the canonical anticommutation relation is given by $\{\hat{\gamma}_i,\hat{\gamma}_j\}=2\delta_{ij}$, an arbitrary operator $\hat{X}\in \mathcal{C}_{2N}$ can be represented as a polynomial in $\{\hat{\gamma}_i\}$ \cite{bravyi_lagrangian_2004}, namely
\begin{equation}
\hat{X}=\alpha\hat{\mathbb{I}} + \sum_{p=1}^{N}\sum_{1 \leq a_1< \cdots < a_p \leq N } \alpha_{a_1,\ldots,a_p}\hat{\gamma}_{a_1} \hat{\gamma}_{a_1+N}\cdots \hat{\gamma}_{a_p}\hat{\gamma}_{a_p+N},
\label{CH3:expantion_of_operators}
\end{equation}
where $\alpha=2^{-N}\operatorname{Tr}(\hat{X})$ and $\alpha_{a_1,\ldots,a_p}$ are the expansion coefficients of $X$. The operator $\hat{X}\in\mathcal{C}_{2N}$ is called even (odd) if it involves only even (odd) powers of the generators $\hat{\gamma}$. Notice that in the expansion of $\hat{X}$ the operators $\hat{\gamma}_{i}$ and $\hat{\gamma}_{i+N}$ are involved, operators that are defined in \eqref{CH2:majorana} through the operators of creation and annihilation at the site $i$. Hence, we can denote by $x_i \in \{0,1\}$ the presence or absence of the operator $\hat{\gamma}_i$. Thus, we can build the binary vector $\vec{x}$ that will give information about the presence or absence of the operator $\hat{\gamma}_i$ in the expansion of $\hat{X}$. Similarly we can construct the vector $\vec{y}$ which will tells us the information about the presence or absence of the operator $\hat{\gamma}_{i+N}$.
Therefore, we have that the expansion of $\hat{X}\in \mathcal{C}_{2N}$, given in \eqref{CH3:expantion_of_operators}, can alternatively be written as
\begin{equation}
\hat{X} = \sum_{\vec{x},\vec{y}} f(\vec{x},\vec{y}) \vec{\gamma}(\vec{x},\vec{y}),
\label{CH3:expantion_operator_X}
\end{equation}
where
\begin{equation}
\vec{\gamma}(\vec{x},\vec{y}) = \hat{\gamma}_{1}^{x_1}\hat{\gamma}_{1+N}^{y_1}\cdots \hat{\gamma}_{N}^{x_N}\hat{\gamma}_{2N}^{y_N},
\end{equation}
where $x_i$ ($y_i$) is $1$ when the operator $\hat{\gamma}_i$ ($\hat{\gamma}_{i+N}$) appears and $0$ otherwise, and $f(\vec{x},\vec{y})$ is the coefficient expansion of $\hat{X}$.  By using this notation, we are able to write products of the form $\vec{\gamma}(\vec{x},\vec{y})\vec{\gamma}(\vec{x}',\vec{y}')$ as
\begin{equation}
\vec{\gamma}(\vec{x},\vec{y})\vec{\gamma}(\vec{x}',\vec{y}') = e^{i\phi(\vec{x},\vec{y},\vec{x}',\vec{y}')} \vec{\gamma}(\vec{x}+\vec{x}',\vec{y}+\vec{y}'),
\label{CH3:relation_vectors}
\end{equation}
where the term $e^{i\phi(\vec{x},\vec{y},\vec{x}',\vec{y}')}$ refers to the phase that appears when dealing with products of Majorana operators, and that in general depends on the sequences $\vec{x},\vec{y},\vec{x}'$ and $\vec{y}'$ (strictly speaking, the phase depends on the involved products of Majorana operators that may appear), and where the sum $\vec{x}+\vec{x}'$ is taken $\mod 2$, since it is not possible to have quadratic terms in any Majorana operator\cite{bravyi_lagrangian_2004}.\\


\indent On the other hand, in the first chapter we saw that ultra-orthogonality appeared as a consequence of typicality, thus ultra-orthogonality has to be studied in the same framework of typicality. It means that the Hilbert subspace in which the states of the universe are sampled has to be constrained to a global restriction (e.g. a defined constant energy $E$). To guarantee that a global restriction is imposed over the states we study, we will make use of a very well-known technique, named Gibbs sampling, for obtaining sequences of observations which are approximated from a specified multivariate probability distribution \cite{robert_multi-stage_2004, gilks_markov_1996, noauthor_gibbs_nodate}. But before we go any further, it is important to translate the concepts of universe, environment and subsystem, defined in chapter one, to the fermionic system of $N$ modes. For the purpose of this work we will identify the universe as the system of $N$ fermionic modes and the subsystem as the portion that corresponds to $L$ fermionic modes, with $L \ll N$.\\

\indent Now consider an excited state $\ket{\vec{n}}$ of the fermionic system, described by the Hamiltonian given in equation \eqref{CH2:QuadraticHamiltonian}. The excited state is described by the set of excitation numbers $\vec{n}\equiv (n_1,n_2,\ldots,n_N)$, where $n_q \in \{0,1\}$. Our goal is then to independently sample the occupations $n_q$ accordingly to the Fermi-Dirac distribution, that is:

\begin{equation}
\operatorname{Prob}(\vec{n} \mid \beta)=\prod_{q=1}^{N} \operatorname{Prob}\left(n_{q} \mid \beta\right), \quad \operatorname{Prob}\left(n_{q} \mid \beta\right)=\frac{e^{-\beta \omega\left(\theta_{q}\right) n_{q}}}{\sum_{n_{q}} e^{-\beta \omega\left(\theta_{q}\right) n_{q}}},
\label{CH3:Gibbs_Sampling}
\end{equation}
where $\theta_q = \frac{2\pi}{N} q$, $\omega(\theta_q)$ is the spectrum of the correspondent of the system described by a Hamiltonian of the form \eqref{CH2:QuadraticHamiltonian} (where in the special case of the $XY$ model, is given by the equation \eqref{CH3:Spectrum_XY_model}), and $\beta$ refers to the inverse temperature. This way of sampling reproduce a well-defined value of the energy when $N\to\infty$, condition that is important to us since the energy will play the role of the global constraint for our system. To see the reason that the energy will be well-defined when $N\to\infty$, we first calculate the average number of excitation in the mode at angle $\theta_q$
\begin{equation}
v\left(\theta_{q} \mid \beta\right) \equiv\left\langle n_{q}\right\rangle_{\beta}=\frac{1}{e^{\beta \omega\left(\theta_{q}\right)} + 1},
\end{equation}
while the variance in the number of excitations is given by
\begin{equation}
\left\langle n_{q}^{2}\right\rangle_{\beta}-\left\langle n_{q}\right\rangle_{\beta}^{2}=\frac{1}{e^{\beta \omega\left(\theta_{q}\right)} + 1}=v\left(\theta_{q}\right)\left(1 - v\left(\theta_{q}\right)\right).
\end{equation}
Thus the mean energy is
\begin{equation}
\langle E\rangle_{\beta}=\sum_{q} v_{q}\left(\theta_{q} \mid \beta\right) \omega_{q}\left(\theta_{q}\right)=N \oint_{N} \frac{d \theta}{2 \pi} f(\theta \mid \beta) \omega(\theta),
\label{CH3:Approx_to_integral}
\end{equation}
where $\oint_{N}$ denotes the Riemman sum approximation to the corresponding integral with $N$ subdivisions. As we are interested in the thermodynamic limit $N\to \infty$, in most cases one may replace the integral for $\oint_N$. Similarly, the energy variance of the sampled states is given by
\begin{equation}
\left\langle\Delta E^{2}\right\rangle_{\beta}=\sum_{q} v\left(\theta_{q}\right)\left(1 - v\left(\theta_{q}\right)\right) \omega_{q}^{2}\left(\theta_{q}\right)=N \oint_{N} \frac{d \theta}{2 \pi} v\left(\theta_{q}\right)\left(1 - v\left(\theta_{q}\right)\right) \epsilon^{2}(\theta).
\end{equation}
Thus, Gibbs sampling provides an even sampling of states within $\Delta E$ of the energy $\langle E\rangle$, where $\Delta E/\langle E\rangle\sim O(N^{-1/2})$.\\

\indent Summarising,we have shown that an operator $\hat{X}\in \mathcal{C}_{2N}$ can be expanded in terms of two sequences $\vec{x}$ and $\vec{y}$ that represent the presence or the absence of the operator $\hat{\gamma}_i$ and $\hat{\gamma}_{i+N}$ respectively. Moreover, we showed that excited states of the fermionic system of $N$ modes fulfilling a condition about the energy, can be sampled independently from the Fermi-Dirac distribution. Having these ideas in mind, we are ready to show our first result.
\section{Studying ultra-orthogonality}
In order to study ultra-orthogonality for the case of systems that can be mapped to quadratic fermionic Hamiltonians, we will consider two randomly chosen, excited states $\ket{\vec{n}_1}$ and $\ket{\vec{n}_2}$. As mentioned in the last part of the chapter one, ultra-orthogonality is a property that whenever we take the partial trace over the environment of the exterior product of two states of the universe ($\ket{\vec{n}_1}$ and $\ket{\vec{n}_2}$) is approximately zero, that is $\operatorname{Tr}_{\mathcal{E}}\ket{\vec{n}_1}\bra{\vec{n}_2}\approx 0$, thus we will be interested in the states of the form: 
\begin{equation}
\hat{\rho}_{12} = \ket{\vec{n}_1}\bra{\vec{n}_2}.
\end{equation}
Thus by taking the partial trace over the $N-L$ sites of the system we wet
\begin{equation}
\hat{X}_{12} = \operatorname{Tr}_{N-L} \hat{\rho}_{12} = \operatorname{Tr}_{N-L} \ket{\vec{n}_1}\bra{\vec{n}_2}.
\end{equation}
Notice that since we took the respective partial trace over the $N-L$ modes, the operator $\hat{X}_{12}$  is an operator that can be expanded in terms of majorana operators of $L$ modes, that is, $\hat{X}_{12}\in \mathcal{C}_{2L}$ and therefore has an expansion in terms of the Majorana operators $\hat{\gamma}_i$ and $\hat{\gamma}_{i+L}$, with $i=1,2,\ldots L$. We then write $\hat{X}_{12}$ with the notation in \eqref{CH3:expantion_operator_X} where we will keep the sequences $\vec{x}$ and $\vec{y}$ to be of length $N$, but for this case we will put zeros in the $N-L$ remaining spaces, thus making explicit the absence of the $N-L$ remaining operators. Thus $\hat{X}_{12}$ will be written as
\begin{equation}
\hat{X}_{12} = \sum_{\vec{x},\vec{y}} f(\vec{x},\vec{y})\vec{\gamma}(\vec{x},\vec{y}),
\label{CH3:x_1_2_state}
\end{equation}
and the sequences $\vec{x},\vec{y}$ are sequences of length $N$ that have zeros in the $N-L$ positions.\\
\indent Thus, our task will be to find the coefficient $f(\vec{x},\vec{y})$ to know the expansion of $\hat{X}_{12}$. In order to do this, we multiply equation  by $\vec{\gamma}^{\dagger}(\vec{x}',\vec{y}')$ and take the partial trace over $L$
\begin{equation}
\operatorname{Tr}_L\left(\hat{X}_{12}\vec{\gamma}^{\dagger}(\vec{x}',\vec{y}')\right)=\sum_{\vec{x},\vec{y}}f(\vec{x},\vec{y}) \operatorname{Tr}_L\left(\vec{\gamma}(\vec{x},\vec{y})\gamma^{\dagger}(\vec{x}',\vec{y}')\right).
\label{CH3:partial_trace_1}
\end{equation}
Using the equation \eqref{CH3:relation_vectors} in \eqref{CH3:partial_trace_1}, the right hand side of the equation will become $\delta_{\vec{x}+\vec{x}'}\delta_{\vec{y}+\vec{y}'}$. Thus the coefficient $f(\vec{x},\vec{y})$ is given by
\begin{equation}
f(\vec{x}',\vec{y}') = \frac{1}{2^L}\operatorname{Tr_L}\left(\hat{X}_{12}\gamma^{\dagger}(\vec{x}',\vec{y}')\right)=\frac{1}{2^L} \bra{\vec{n}_2}\gamma^{\dagger}(\vec{x}',\vec{y}')\ket{\vec{n}_1}.
\label{CH3:my_relation_coefficients}
\end{equation}
So in order to find the coefficient $f(\vec{x},\vec{y})$ we have to explicitly show how the operator $\vec{\gamma}(\vec{x},\vec{y})$ acts over $\ket{\vec{n}_1}$ ($\ket{\vec{n}_2}$). To explicitly show this, we recall from equation \eqref{CH2:Williamson_Cov_fermionic_matrix} that there is an orthogonal transformation that brings the Majorana operators acting on the spatial modes of fermionic states to the normal modes, that is
\begin{equation}
\overbrace{\gamma_{i_1}\gamma_{i_2}\ldots\gamma_{i_k}}^{\text{Spacial modes}}= O_{i_1\alpha_1}O_{i_2\alpha_2}\ldots O_{i_k\alpha_k} \underbrace{\gamma_{\alpha_1}\gamma_{\alpha_2}\ldots\gamma_{\alpha_k}}_{\text{Normal modes}}.
\label{CH3:transformation_to_normal_modes}
\end{equation}
where the $O_{i_j,\alpha_j}$ refers to the orthogonal transformation discussed in section $2.3$. Now, notice that the operators $\vec{\gamma}$ obtained after this orthogonal transformation will be diagonal over the sequences $\vec{x}$, $\vec{y}$. This implies that the are two possible ways of getting an excited state $\ket{\vec{n}_1}$ through the function $\vec{\gamma}$ that is written on the normal mode basis. The first one is by applying the the operator $\vec{\gamma}(\vec{n}_1,0)$ over the vacuum state $\ket{0}$, that is
\begin{equation}
\ket{n_1}=\vec{\gamma}(\vec{n}_1,0)\ket{0},
\end{equation}
and the second is
\begin{equation}
\ket{n_1}=\vec{\gamma}(0,\vec{n}_1)\ket{0}e^{i\phi(\vec{n_1})}.
\end{equation}
So, the $\vec{x}$ sequence takes the vacuum state $\ket{0}$ and turns it into an excited state by changing the zeros to ones, and alternatively the $\vec{y}$ sequence transforms the vacuum state $\ket{0}$ to an excited state and adds a phase.\\
\indent Therefore, the right hand-side of equation \eqref{CH3:my_relation_coefficients} yields
\begin{equation}
\bra{\vec{n}_2}\vec{\gamma}(\vec{x},\vec{y})\ket{\vec{n}_1}\propto \delta_{\vec{n}_1+\vec{x}+\vec{y},\vec{n}_2}e^{i\phi(\vec{n}_1,\vec{n}_2,\vec{x},\vec{y})} = \delta_{\vec{x}+\vec{y},\vec{n}_2+\vec{n}_1}e^{i\phi(\vec{n}_1,\vec{n}_2,\vec{x},\vec{y})},
\end{equation} 
where the last equality is guaranteed since the sums are taken to be modulo $2$.\\
\indent The coefficient $f(\vec{x},\vec{y})$, is then given by
\begin{equation}
f(\vec{x},\vec{y}) = \frac{1}{2^L}\sum_{\vec{x}',\vec{y}'} \mathcal{U}_{\vec{x}\vec{x}'} \mathcal{V}_{\vec{y}\vec{y}'} \underbrace{\bra{\vec{n}_2}\vec{\gamma}(\vec{x},\vec{y})\ket{\vec{n}_1}}_{\propto \delta_{\vec{n}_1+\vec{n}_2,\vec{x}+\vec{y}}}.
\label{CH3:my_Relation_coeffitients_final}
\end{equation}
where $\mathcal{U}_{\vec{x}\vec{x}'}$ ($\mathcal{V}_{\vec{y}\vec{y}'}$) refers to the orthogonal matrix transformation that brings the operators $\gamma_{i}$ ($\gamma_{i+L}$) to its normal mode representation. Note that the vector $\vec{n}_1+\vec{n}_2$ is the vector that has a one in the $i-$th component when both vectors ($\vec{n}_1$ and $\vec{n}_2$) differ in the position $i$, that is, $n_{1_i}\neq n_{2_i}$. Thus, the vector $\vec{n}_1+\vec{n}_2$ is interpreted as the error vector. of the sequences\\

\indent The result in equation \eqref{CH3:my_Relation_coeffitients_final} is extremely important because it shows that when we work with states like $\hat{X}_{12}$, if the error vector $\vec{n}_1+\vec{n}_2$ has a larger number of ones than the vector $\vec{x}+\vec{y}$, the coefficient $f(\vec{x},\vec{y})$ will immediately become zero. Moreover, notice that we have defined $\vec{x}$ and $\vec{y}$ to be sequences that allows us to expand the operator $\hat{X}_{12}\in\mathcal{C}_{2L}$, meaning that the vector $\vec{x}+\vec{y}$ can have at most $L$ ones. Therefore, we have that the coefficient $f(\vec{x},\vec{y})$ will be equal to zero when the number of ones in the vector $\vec{n}_1+\vec{n}_2$ exceeds $L$.\\
\indent This particular result brings questions like, ``how likely is it to have more than $L$ error when $N\gg L$?'' and ``what happens when we have less errors than $L$?''. In next section we will be addressing the first question, namely, we will interpret the result found in equation \eqref{CH3:my_Relation_coeffitients_final} in terms of a random minimum distance code and we will show that the largest Hilbert subspace in which ultra-orthogonality holds is exponentially large.
\section{Fermionic random minimum codes.}
In order to make an explicit connection between minimum distance codes, described in the last part of the second chapter, and the found result in the previous section, consider the \textit{error vector} $\vec{e}_{12}$ to be the sum (modulo $2$) of the sequences $\vec{n}_1$ and $\vec{n}_2$
\begin{equation}
\vec{e}_{12} = \vec{n}_1 + \vec{n}_2.
\end{equation} 
As we pointed out before, the vector $\vec{e}_{12}$ has ones in the entries where $\vec{n}_1$ and $\vec{n}_2$ differ, that is
\begin{equation}
e_{12_{i}}= \begin{cases} 0\qquad n_{1_i}=n_{2_i},\\
1 \qquad n_{1_i}\neq n_{2_i}.
 \end{cases}
\end{equation}
We denote by $d$ the Hamming distance between the sequences $\vec{n}_1$ and $\vec{n}_2$, or alternatively by the Hamming weight of $\vec{e}_{12}$, that is
\begin{equation}
d=d(\vec{n}_1,\vec{n}_2) = w(\vec{e}_{12}).
\end{equation}
In the previous section showed that there is a specific distance $d>L$ at which the operator $\hat{X}_{12}$, given by equation \eqref{CH3:x_1_2_state}, is equal to zero. Therefore, we can write the problem of working with states $\ket{\vec{x}^{(i)}}$ in terms of a $(N,M,d)$ code $\mathfrak{C}$ over the alphabet $F=\{0,1\}$ and the channel $S$ given by the memoryless binary symmetric channel. Thus, the code $\mathfrak{C}$ is a binary code with code length $N$ and code size $M$. Denoting the information length of the code $\mathfrak{C}$ as $k=\log_2 M$, and the rate $R=k/N$, we have that the code $\mathfrak{C}$ will be form by the codewords
\begin{equation}
\mathfrak{C}=\{\mathbf{x}^{(1)},\mathbf{x}^{(2)},\ldots,\mathbf{x}^{(2^k)}\} \quad \mathbf{x}^{(i)}\in\{0,1\}^{N}.
\end{equation}
thus if we assign each of the codewords $\mathbf{x}^{(i)}$ to the respective state $\ket{\vec{x}^{(i)}}$ such that the states $\{\ket{\vec{x}^{(1)}},\ket{\vec{x}^{(2)}},\ldots\}$ are mutually ultra-orthogonal, we can ask ourselves about the Hilbert subspace $\mathcal{H}_{\mathfrak{C}}$ spanned by the states $\ket{\vec{x}^{(i)}}$, that is
\begin{equation}
\mathcal{H}_{\mathfrak{C}} = \operatorname{Span}\left( \ket{\vec{x}^{(1)}},\ket{\vec{x}^{(2)}},\ldots,\ket{\vec{x}^{(2^k)}} \right).
\end{equation}
Notice that by assigning the codewords $\mathbf{x}^{(i)}$ to the respective state $\ket{\vec{x}^{(i)}}$, the condition of ultra-orthogonality is guaranteed since the code $\mathcal{C}$ is a minimum distance code $(N,M,d)$ with $d>L$, and as we pointed out in chapter two, a minimum distance code has the property that the minimum Hamming distance of any two distinct codewords of $\mathfrak{C}$ is $d$
\begin{equation}
d=\min_{\mathbf{x}^{(i)},\mathbf{x}^{(j)}\in \mathfrak{C}: \mathbf{x}^{(i)}\neq\mathbf{x}^{(j)} } d(\mathbf{x}^{(i)},\mathbf{x}^{(j)}).
\end{equation}
Now take an arbitrary state $\ket{\psi(t)}\in\mathcal{H}_{\mathfrak{C}}$. We can expand it in terms of the elements of its basis as
\begin{equation}
\ket{\psi(t)} = \sum_{\vec{n}\in\mathfrak{C}}\psi(\vec{n},t)\ket{\vec{n}}.
\end{equation}
Defining $\hat{\rho} =\ket{\psi(t)} \bra{\psi(t)}$ the state of the system with $N$ modes, we have that by construction of the Hilbert subspace $\mathcal{H}_{\mathfrak{C}}$, when we take the partial trace over the $N-L$ fermionic modes, we will end up with a time-independent state, that is
\begin{equation}
\rho_{L} = \operatorname{Tr}_{N-L}\left(\ket{\psi(t)}\bra{\psi(t)}\right) = \sum_{\vec{n}\in\mathfrak{C}} |\psi(\vec{n})|^2  \operatorname{Tr}_{N-L}\ket{\vec{n}}\bra{\vec{n}},
\end{equation}
where we have ignored the crossed terms 
\begin{equation}
 \sum_{\vec{n}\neq\vec{m}}\psi(\vec{n},t)\psi(\vec{m},t)^{*} \operatorname{Tr}_{N-L} \ket{\vec{n}}\bra{\vec{m}}.
\end{equation}
since by construction they are all zero ($\operatorname{Tr}_{N-L} \ket{\vec{n}}\bra{\vec{m}}=0$).\\
\indent Thus, every state $\ket{\psi(t)}\in\mathcal{H}_{\mathfrak{C}}$ of the system of $N$ fermionic modes has the property that its reduce state is instantaneously stationary, condition that we can write as
\begin{equation}
\frac{d\rho_L}{dt}=0.
\end{equation}

Having this in mind, we can ask ourselves about the largest minimum distance code $(N,M,d)$ that can be built to alternatively answer what is the largest Hilbert subspace where ultra-orthogonality holds.\\
\indent In order to do this we will follow thoroughly the steps of section $2.5.6$ to compute the average number of codeword that are at a given distance $d$. Note that the probability that two codewords differ in $d$ sites was computed in the last part of the second chapter. However, the probability was computed by supposing that each entry had equal probability to occur, and this is not the case for excited fermionic states, since as we saw in equation \eqref{CH3:Gibbs_Sampling}, each entry of the sequence has a different probability to occur and it depends on the mode $q$. Therefore, the probability that two sequences differ in $d$ sites has to be modified for the case of excited states.\\
\indent Note that the probability that at position $q$ the entry is given by the Fermi-Dirac distribution. Thus the probability $p(\theta_q)$ that two sequences differ in position $q$ will be given by
\begin{equation}
p(\theta_q) = {2\choose 1}\operatorname{Prob}(n_{q}|\beta)\left(1- \operatorname{Prob}(n_{q}|\beta)\right),
\end{equation}
where $\operatorname{Prob}(n_{q}|\beta)$ corresponds to the Fermi-Dirac distribution in equation \eqref{CH3:Gibbs_Sampling}.\\
\indent If we define the random variable $X_i(\theta_q)$ to take the value $1$ with probability $p(\theta_q)$ and $0$ with probability $1-p(\theta_q)$. Explicitly that is
\begin{equation}
\operatorname{Prob}(X_i(\theta_q)= k) = \begin{cases}
p(\theta_q )\qquad &k=1,\\
1-p(\theta_q) \qquad &k=0.
\end{cases}
\end{equation}
Define the sum of the random variables $X=\sum_{k}X_i(\theta_q)$. Notice that $X$ will be nothing but a random variable that equals the total number of errors between two sequences.\\
\indent We will be interested to see what is the probability that the random variable $X$ is equal or larger than $d$, that is
\begin{equation}
\operatorname{Prob}(X\geq d),
\label{CH3:probability_to_bound}
\end{equation}
To compute this probability is in general a complicated task and involve complex expression; instead, we will bound the probability in equation \eqref{CH3:probability_to_bound}. To do this we will make use of the independence of the random variables $X_{i}(\theta_q)$, and a well-know bound defined in equation \eqref{CH2_chernoff_bound}, named \textit{Chernoff bound}.\\
Let $S$ be a real random number grater than zero ($S>0$), then from equation \eqref{CH2:proof_chernoff_bound} we know that
\begin{equation}
\operatorname{Prob}(X\geq d) = \operatorname{Prob}(e^{SX}\geq e^{Sd})\leq \left\langle e^{S X}\right\rangle e^{-S d},
\label{CH3:chernoff_bound_general}
\end{equation}
where $ \left\langle e^{S X}\right\rangle$ refers to the expected value.\\
\indent Particularly we are interested in the optimal $S$, then, as we pointed out in equation \eqref{CH2:optimal_chernoff_bound}, this optimization can be done by taking the minimum $S$ in inequality \eqref{CH3:chernoff_bound_general}
\begin{equation}
\operatorname{Prob}(e^{SX}\geq e^{Sd})\leq \min_{S>0} \left\langle e^{S X}\right\rangle e^{-S d}.
\label{CH3:optimizing_s_chernoff}
\end{equation}
We then have to compute the expected value $\left\langle e^{S X}\right\rangle$ in order to proceed with the optimization. Since $X$ is the sum of random independent variables we have that
\begin{equation}
\left\langle e^{S X}\right\rangle = \prod_{q}  E(e^{S X(\theta_q})) = \prod_q \left(1+p(\theta_k)(e^S -1)\right) = e^{\sum_{q}\log(1+p(\theta_q)(e^S -1))}.
\label{CH3:Bound_to_X}
\end{equation}
Thus, putting equation \eqref{CH3:optimizing_s_chernoff} and \eqref{CH3:Bound_to_X}, we get
\begin{equation}
\operatorname{Prob}(e^{SX}\geq e^{Sd})\leq \min_{S>0}  e^{\sum_{q}\log(1+p(\theta_q)(e^S -1)-Sd)}.
\label{CH3:result_bound}
\end{equation}
Defining $r(\delta)$ as the minimization exponent in equation \eqref{CH3:result_bound}, we get
\begin{equation}
r(\delta) = \min_{S} \frac{1}{N} \left(\sum_{q}\log(1+p(\theta_q)(e^S -1) - S\delta\right),
\label{CH3:error_exponent}
\end{equation}
where $\delta = d/N$. We are interested in the case $N\to \infty$,  meaning that $r(\delta)$ can be written as
\begin{equation}
r(\delta)=\min_{S}\oint_{N} \frac{d\theta}{2\pi}\log\left(1+p(\theta)(e^S -1)\right) -S\delta,
\label{CH3:random_code_exponent}
\end{equation}
where $\oint_{N}$ is the one defined in equation \eqref{CH3:Approx_to_integral}. Note that $r(\delta)$ corresponds to the random-coding exponent associated with the fermionic random states, which we discussed in equation \eqref{CH2:randomexpoenent}. For these reason we refer to the binary sequence of excitations present in a fermionic state as a \textit{fermionic random code}.\\
\indent Even though it is not possible to analytically solve equation \eqref{CH3:random_code_exponent} for $S$, we can differentiate it with respect to $S$ to get a transcendental equation
\begin{equation}
\delta = \oint_N \frac{d\theta}{2\pi} \frac{p(\theta)e^{S}}{1-p(\theta) + p(\theta)e^S}.
\label{CH:optimizing_code_exponent}
\end{equation}
By using equation \eqref{CH3:random_code_exponent} we have that after the optimization over $S$, the inequality in \eqref{CH3:chernoff_bound_general} takes its minimum value \cite{barg_random_2002}. Therefore, if similarly as in section $2.5.6$, we ask ourselves about the number of unordered pairs of codewords $(\mathbf{x}^{(i)},\mathbf{x}^{(j)})$ in $\mathfrak{C}$ with $i\neq j$, at a Hamming distance $d$ apart ($\mathcal{S}_{\mathcal{C}}$). As we showed in equation \eqref{CH2:average_number_codes}, the average number of unordered pairs of codewords  at a distance $d$ apart is given by
\begin{equation}
\langle S_{\mathfrak{C}}(d)\rangle = {M \choose 2} 2^{-Nr(\delta)}.
\label{CH3:average_of_S}
\end{equation}
Remembering that $k=\log_{2} M$ and $R=k/N$, we can replace $M= 2^{NR}$ in equation \eqref{CH3:average_of_S} and conclude
\begin{equation}
\langle S_{\mathcal{C}}(d)\rangle =  2^{N(2R-r(\delta))}.
\label{CH3:average_number_of_S_final}
\end{equation}
From this expression we conclude that whenever we work with rates higher than $r(\delta)/2$, the average number of unordered pairs of codewords at a distance $d$ is exponentially large with the number of modes $N$ in the fermionic system. And conversely, whenever we are at lower rates than $r(\delta)/2$, this number goes to zero exponentially.\\
With the result shown in equation \eqref{CH3:average_number_of_S_final} we conclude that when the rate of the random fermionic code is larger than rates higher than $r(\delta)/2$, there is an associated exponentially large Hilbert subspace $\mathcal{H}_{\mathfrak{C}}$ in which every state taken from it has the property that its correspondent reduced state is automatically stationary, in other words, we are able to find exponentially large Hilbert subspaces in which the property of ultra-orthogonality holds exactly.