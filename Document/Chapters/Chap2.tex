\chapter{Characterization of Fermionic Gaussian States}

\section*{Overview}
In many areas of physics one has to has to deal with solving quantum many body problems, which is often a computationally difficult if not impossible task. It is also well known that under appropriate approximations a wide class of complicated Hamiltonians with many-body interactions can be often be mapped onto Hamiltonians that are quadratic in annihilation and creation operators and have the generic form \cite{botero_bcs-like_2004}

\begin{equation}
\hat{H}=\sum_{i j} C_{i j} \hat{a}_{i}^{\dagger} \hat{a}_{j}+\sum_{i j}\left(A_{i j} \hat{a}_{i}^{\dagger}\hat{a}_{j}^{\dagger}+\mathrm{h.c.}\right)
\label{CH2:QuadraticHamiltonian}
\end{equation}

where $i,j$ run from $1$ to the number of modes in the system ($N$) and $\hat{a}_i$, $\hat{a}^{\dagger}_i$ are Fermionic annihilation and creation operators which satisfy the canonical anti-commutation relations (CAR)\cite{fradkin_field_1997}

\begin{equation}
\left\{\hat{a}_{k}, \hat{a}_{l}\right\}=\left\{\hat{a}_{k}^{\dagger}, \hat{a}_{l}^{\dagger}\right\}=0, \quad\left\{\hat{a}_{k}, \hat{a}_{l}^{\dagger}\right\}=\delta_{k l}.
\label{CH2:Anticommutation}
\end{equation}

A convenience when working with these kind of Hamiltonians is that can diagonalized via a Bogoliubov- Valantin transformations transformation (i.e., canonical transformations), which maps Fermionic creation and annihilation operators on the creation and annihilation operators of non-interacting quasi-particles\cite{berezin_method_1966,bogoljubov_new_1958}. Explicitly the transformation looks like

\begin{equation}
\begin{array}{c}
\hat{a}_{i} \mapsto \gamma_{i} \hat{q}_{i}+\kappa_{i} \hat{q}_{i}^{\dagger}, \\
\hat{a}_{i}^{\dagger} \mapsto \bar{\gamma}_{i} \hat{q}_{i}^{\dagger}+\bar{\kappa}_{i} \hat{q}_{i}.
\end{array}
\label{CH2:Bogoliuvov}
\end{equation}

where $\gamma_i , \kappa_i$ are complex numbers such that preserves the canonical anti-commutation relations given by \eqref{CH2:Anticommutation} for $\hat{q}$, $\hat{q}^{\dagger}$\footnote{This relation can also be expresses as a condition over $\gamma_i , \kappa_i$,
\[ \gamma_i ^2+ \kappa_i^2 = 1,\]
and 
\[\left\{\hat{q}_{k}, \hat{q}_{l}\right\}=\left\{\hat{q}_{k}^{\dagger}, \hat{q}_{l}^{\dagger}\right\}=0, \quad\left\{\hat{q}_{k}, \hat{q}_{l}^{\dagger}\right\}=\delta_{k l}.\]
 }.
\\
Many relevant physics models are diagonalizable via a Bogoliubov-Valantin transformations, some examples are the Hubbard model, BCS theory of superconductivity in the mean field or Hartree-Fock approximation, and certain solvable spin-chain models (After a Jordan-Wigner transformation) \cite{fradkin_field_1997}. As we will later explain, an important feature about the class of Hamiltonians described by \eqref{CH2:QuadraticHamiltonian}, if not the most important for the purpose of this project,  is that not only the ground state (quasi-particle vacuum)  but all eigenstates describing a excitation  in a set of quasi-particles, belong to the class of so-called Fermionic Gaussian states\cite{botero_bcs-like_2004}. What is important about these class of states is that are fully characterized by second order correlations, because all the higher moments factorize. This is known as Wick's theorem\cite{bravyi_classical_2005, molinari_notes_2017}, being this the motivation to formulate an equivalent and convenient characterization of second order correlations, which is defined in terms of Majorana fermions and is known as Fermionic covariance matrix.
\section*{Majorana Fermions}
Majorara fermions are fermions such that they are their own antiparticle. In the frame of condensed matter Majorana Fermions (quasi-particles) can be interpreted as a superposition of a electron state and a hole \cite{leijnse_introduction_2012}.
\\
The formalism of Majorana fermions or Majorana modes of the system (with $N$ modes) can be introduce with the operators
\begin{equation}
\hat{c}_{2j-1}=\hat{a}_{j}^{\dagger}+\hat{a}_{j}, \quad \hat{c}_{2j}=(-i)\left(a_{j}^{\dagger}-a_{j}\right).
\label{CH2:majorana}
\end{equation}
In which case its canonical anti-commutation relations (CAR) take the form
\begin{equation}
\left\{\hat{c}_{k},\hat{c}_{l}\right\}=2 \delta_{k l}.
\label{CH2:CAR_majorana}
\end{equation}

The anti-commutation relations is seen to be a consequence of an $\mathbb{R}^{2N}$ Clifford algebra\footnote{By inspection of \eqref{CH2:CAR_majorana} we see that  any linear transformation of the form $\tilde{\gamma}_{\alpha} = O_{\alpha\beta}\gamma_{\beta}$ where $O\in SO(2N)$, the special orthogonal group in $2N$  dimensions}. Transformation in between Fermionic and Majorana operators is achieved by matrix of the block form
\begin{equation}
\Omega=\left(\begin{array}{cc}
\mathbb{I} & \mathbb{I} \\
i \mathbb{I} & -i \mathbb{I}
\end{array}\right),
\end{equation}

and then the map from Fermionic ($\vec{\hat{a}}^{T} = (\hat{a}_1,\ldots,\hat{a}_1^{\dagger},\ldots)$) and Majorana ($\vec{\hat{c}}^{T} = (\hat{c}_1,\ldots,\hat{c}_1^{\dagger},\ldots)$) operators is written as $\Omega\vec{\hat{a}}=\vec{\hat{c}}$.
\\
By changing from Fermionic operators to Majorana operators is possible, and convenient, to define de Fermionic covariance matrix which fully characterise Gaussians states\footnote{In comparison to its boson counterpart the fermion Gaussian states have the property that correlation functions for the creation/annihilation operators are completely determined by the two-point functions according to Wick’s theorem \cite{westwanski_general_1973}, and moreover,  since this property is extensible to correlation function pertaining to a reduced subset of the modes, it follows that any partial (reduced) density matrix obtained from $\rho$ remains Gaussian.}.
\section*{Fermionic Covariance matrix }
A system of $N$ fermion modes, described by a set of creation and annihilation operators $\hat{a}^{\dagger}, \hat{a}$  and satisfying the  canonical anti-commutations relations in \eqref{CH2:Anticommutation}, is Gaussian if for such system any state $\rho$ can be written as  \cite{cheong_many-body_2003}

\begin{equation}
\rho=\bigotimes_{k=1}^{N} \tilde{\rho}_{k}, \quad \tilde{\rho}_{k}=\frac{1}{2}\left(1-\lambda_{i}\left[\tilde{a}_{i}^{\dagger}, \tilde{a}_{i}\right]\right),
\label{CH2:rho_gaussiano_no_exp}
\end{equation}

for a certain choice of mode basis $\tilde{a}=u_{i}^{j}\hat{a}_{j} + v_{i}^{j}\hat{a}_{j}^{\dagger}$, and with $|\lambda_i|\leq 1$, where the equality holds for pure states. Equivalently, as mentioned before, Gaussian States are fully characterized by their second moments, so an equivalent form of writing $\rho$ is
\begin{equation}
\rho= \frac{1}{Z}\cdot \exp \left[-\frac{i}{4} \hat{c}^{T} G \hat{c}\right],
\label{CH2:rho_gaussiano_exp}
\end{equation}
with $\hat{c} = (\hat{c}_1,\hat{c}_2,\ldots,\hat{c}_{2N})$, the vector of Majorana operators \eqref{CH2:majorana}, $Z$ a normalization constant and $G$ real anti-symmetric $2N\times 2N$ matrix. 
\\
Since $G$ is a skew-symmetric matrix, it can always be brought to the block diagonal form 
\begin{equation}
O G O^{T}=\bigoplus_{i=1}^{N}\left(\begin{array}{cc}
0 & -\beta_{j} \\
\beta_{j} & 0
\end{array}\right) \quad \text { with } \quad O \in \mathrm{SO}(2 \mathrm{N}),
\label{CH2:MatrixG_Williamson}
\end{equation}
by a special orthogonal matrix $O\in SO(2N)$ where the $\beta_{j}$ are called the Williamson eigenvalues of the matrix $G$. 
From equation \eqref{CH2:rho_gaussiano_exp} it is clear that Gaussian states have an interpretation as thermal (Gibbs) states corresponding to a Hamiltonian of the form
\begin{equation}
\hat{H}=\frac{i}{4} \hat{c}^{T}G\hat{c}= \frac{i}{4} \sum_{k>l}G_{kl}\left[\hat{c}_{k},\hat{c}_{l}\right].
\label{CH2:Hamiltonian_majorana}
\end{equation}

and \eqref{CH2:MatrixG_Williamson} shows that every Gaussian state has a normal mode decomposition in terms of $N$ single mode ``thermal states'' of the form \eqref{CH2:rho_gaussiano_no_exp} ($\sim \text{exp}(-\beta \hat{a}^{\dagger}\hat{a})$)\cite{kraus_pairing_2009}. From this is clear, that the state can be fully determined by the expectation values of quadratic operators $(\hat{a}_i^{(\dagger)}\hat{a}_j^{(\dagger)}$ and  $\hat{a}^{\dagger}_i\hat{a}_j)$. So collecting these expectation values in a real and skew-symmetric covariance matrix $\Gamma$ which is defined via
\begin{equation}
\Gamma_{k l}=\frac{i}{2} \operatorname{tr}\left(\rho\left[c_{k}, c_{l}\right]\right).
\label{CH2:Cov_matrix_elements}
\end{equation}
Since $\Gamma$ is also a anti-symmetric matrix, it can be brought into a block diagonal form by a canonical transformation
\begin{equation}
\tilde{\Gamma} = O \Gamma O^{T}=\bigoplus_{i=1}^{M}\left(\begin{array}{cc}
0 & \lambda_{i} \\
-\lambda_{i} & 0
\end{array}\right),
\label{CH2:Williamson_Cov_fermionic_matrix}
\end{equation}
with $\lambda_i\geq 0$ the Williamson eigenvalues.\\
In terms of the creation/annihilation operators obtained from the transformed $\tilde{c}=O\hat{c}$, the Gaussian state $rho$ takes the form \eqref{CH2:rho_gaussiano_exp} with $\tilde{\Gamma}$ as its Fermionic covariance matrix. It is easy to see that the relation between $G$ and $\Gamma$ is given by $\lambda_{i} = \tanh{\left(\beta_{i}/2\right)}$, for $i=1,2,\ldots,N$\cite{kraus_pairing_2009}. 
\newline
The equivalence between the special orthogonal group in $2N$ dimensions ($SO(2N)$) and the Fermionic Gaussian states, drives to interesting an interesting  property about states describing multi-particles excitations. If $\ket{\textit{vacuum}}$ is the ground state of some Hamiltonian, with annihilation operators $\hat{a}_{i}$ in a given quasi-particle basis, then $\hat{a}_{i}^{\dagger}\ket{\textit{vacuum}} = \hat{c}_{2i}\ket{\textit{vacuum}}$. Meaning that any multi-particle state of this kind is obtained from some transformation (that preserves the canonical anti-commutation relations) of the ground state $\ket{\textit{vacuum}}$, remains Gaussian. In other words, Gaussian states are preserved under any unitary transformation that preserves anti-commutation relations.
\newline 
The fact that all eigenstates of the Hamiltonian in \eqref{CH2:QuadraticHamiltonian} are Gaussian as well as the fact that this property is extensible to a reduced subset of the modes, is quite important since a big part of this work is focused on the study of the Fermionic covariance matrix of the $XY$ model.
\section{Grassman Approach}











