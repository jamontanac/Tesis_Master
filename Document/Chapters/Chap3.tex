\chapter{Getting to understand Ultra Orthogonality in the $XY$ Model.}

As discussed in the previous chapters, we are interested in solvable Fermionic systems. Indeed the one dimensional $XY$ model is one of those systems\cite{lieb_two_1961}. The XY Hamiltonian model is a set of $N$ spin $1/2$ particles 
located on the sites of d-dimensional lattice. Nevertheless, whenever we talk about the $XY$ model, we will be referring to the $1D$ $XY$ model.
\section{The $XY$ Model}
A chain of $N$ spins where each spin is able to interact with its nearest neighbours in the $X$ and $Y$ coordinate as well as an external magnetic field, will be described by the Hamiltonian of the form
\begin{equation}
H_{X Y}=-\frac{1}{2} \sum_{l=0}^{N-1}\left(\frac{1+\gamma}{2} \sigma_{l}^{x} \sigma_{l+1}^{x}+\frac{1-\gamma}{2} \sigma_{l}^{y} \sigma_{l+1}^{y}+\lambda \sigma_{l}^{z}\right),
\label{CH3:Hamiltonian_XY}
\end{equation}
where $\gamma$ is so-called the anisotropy parameter and represents the difference between the strength of the $XX$ interaction and the $YY$ interaction\footnote{When we talk about interactions we mean interactions between spins.}, $\lambda$ is the intensity of the external magnetic field and $\sigma^{i}_{l}$ is the Pauli matrix ($i= x,y,z$) acting over the $l$ site of the chain.\\
The XY model is a model that has been widely studied for a variety of values of $\lambda$ and $\gamma$ and in some limits it has a correspondence to other models of interest in condensed matter\cite{katsura_statistical_1962,barouch_statistical_1971,barouch_statistical_1970}\footnote{Some examples of this kind, are the boson Hubbard model in the limit of hard bosons. the case when $\gamma=1$ correspond to the Ising model, and the Kitaev chain is equivalent to the XY model under a proper identification of the parameters $\mu$, $t$ and $\Delta$ with $\gamma$ and $\lambda$\cite{katsura_statistical_1962,barouch_statistical_1971}}.
\subsection{The spectrum}
To find the spectrum of the Hamiltonian \eqref{CH3:Hamiltonian_XY} we need to perform some special transformations.
\subsection{Jordan-Wigner Transformation}
We first consider the non local transformation given by
\begin{equation}
\hat{b}_{l}=\left(\prod_{m<l} \sigma_{m}^{z}\right) \sigma_{l}^{-}, \quad \sigma_{l}^{-}=\frac{\sigma_{l}^{x}-i \sigma_{l}^{y}}{2},
\end{equation}
where these $b_l$ represent spinless Fermionic operators, and its canonical anticommutation relation (CAR) is given by\cite{reyes-lega_aspects_2016} 
\begin{equation}
\left\{\hat{b}_{i}^{\dagger}, \hat{b}_{j}^{\dagger}\right\}=\left\{\hat{b}_{i}, \hat{b}_{j}\right\}=0, \quad\left\{\hat{b}_{i}^{\dagger}, \hat{b}_{j}\right\}=\delta_{i, j}.
\end{equation}
So inverting the transformation we get 
\begin{equation}
\begin{array}{l}
\sigma_{l}^{z}=1-2 \hat{b}_{l}^{\dagger} \hat{b}_{l} \\
\sigma_{l}^{x}=\left(\prod_{m<l}\left(1-2 \hat{b}_{m}^{\dagger} \hat{b}_{m}\right)\right)\left(\hat{b}_{l}^{\dagger}+\hat{b}_{l}\right) \\
\sigma_{l}^{y}=i\left(\prod_{m<l}\left(1-2 \hat{b}_{m}^{\dagger} \hat{b}_{m}\right)\right)\left(\hat{b}_{l}^{\dagger}-\hat{b}_{l}\right).
\end{array}
\end{equation}

The terms of interaction in the Hamiltonian will look as
\begin{equation}
\begin{aligned}
\hat{\sigma}_{l}^{x} \hat{\sigma}_{l+1}^{x} &=\left(\hat{b}_{l}^{\dagger}-\hat{b}_{l}\right)\left(\hat{b}_{l+1}^{\dagger}+\hat{b}_{l+1}\right) \\
\hat{\sigma}_{l}^{y} \hat{\sigma}_{l+1}^{y} &=-\left(\hat{b}_{l}^{\dagger}+\hat{b}_{l}\right)\left(\hat{b}_{l+1}^{\dagger}-\hat{b}_{l+1}\right)
\end{aligned},
\end{equation}
and the Hamiltonian will look like,
\begin{equation}
H_{X Y}=-\frac{1}{2} \sum_{l}\left[\left(\hat{b}_{l+1}^{\dagger} \hat{b}_{l}+\hat{b}_{l}^{\dagger} \hat{b}_{l+1}\right)+\gamma\left(\hat{b}_{l}^{\dagger} \hat{b}_{l+1}^{\dagger}-\hat{b}_{l} \hat{b}_{l+1}\right)\right]-\frac{\lambda}{2} \sum_{l}\left(1-2 \hat{b}_{l}^{\dagger} \hat{b}_{l}\right),
\end{equation}
after this transformation. The term of $-\lambda N/2$ is usually ignored since it cause only a gauge of the spectrum in the energy\cite{reyes-lega_aspects_2016} .
\subsection{Fourier Transformation}
It is possible to exploit an other symmetry in the system. It comes by considering periodic boundary conditions (PBC)\cite{reyes-lega_aspects_2016} . This can be easily done by identifying the spin in the site $N$ with the spin in the site $1$. After imposing this condition, we have that the Fourier transform of the operator $\hat{b}_l$ will look as
\begin{equation}
\hat{d}_{k}=\frac{1}{\sqrt{N}} \sum_{l=1}^{N} \hat{b}_{l} e^{-i \phi_{k} l},
\end{equation}
with $\theta_{k}=\frac{2 \pi}{N} k$.\\
Since the Fourier transformation is unitary, the operators $\hat{d}_k$ will preserve the CAR.
\begin{figure}[H]
    \centering
    \includegraphics[width=0.4\textwidth]{Figures/Periodic_boundaries.png}
    \caption{Illustration of what a boundary condition means in the case of our spin chain}
    \label{periodic condition}
\end{figure}
Thus, the Hamiltonian can be written in terms of the operators $\hat{d}_k$ as
\begin{equation}
H_{X Y}=\sum_{k=-(N-1) / 2}^{(N-1) / 2}\left(-\lambda+\cos \phi_{k}\right) \hat{d}_{k}^{\dagger} \hat{d}_{k}+\frac{i \gamma}{2} \sum_{k=-(N-1) / 2}^{(N-1) / 2} \sin \phi_{k}\left(\hat{d}_{k} \hat{d}_{-k}+h . c\right),
\end{equation}
where we have ignored an additional term which is proportional to $1/N$ which will vanish for in the thermodynamic limit $N\to \infty$\cite{barouch_statistical_1970,barouch_statistical_1971}, which is our case of interest.
\subsection{Bogoliubov -Valantin Transformation}
As mentioned before Fermionic quadratic Hamiltonians can be easily diagonalised via a Bogoliubov-Valantin transformation over the operators $\hat{d}_k$

\begin{equation}
\tilde{d}_{k}=u_{k} \hat{d}_{k}^{\dagger}+i v_{k} \hat{d}_{-k}.
\end{equation}
Since we want this transformation to preserve CAR, it is needed that $u_k^2 + v_k^2 = 1$, which implies that we can use the parametrization $u_{k}=\cos \left(\psi_{k} / 2\right)$ and $v_{k}=\sin \left(\psi_{k} / 2\right)$, with
\begin{equation}
\cos \frac{\psi_{k}}{2}=\frac{-\lambda+\cos \phi_{k}}{\sqrt{\left(\lambda-\cos \phi_{k}\right)^{2}+\left(\gamma \sin \phi_{k}\right)^{2}}}
\end{equation}
So finally our Hamiltonian will look as
\begin{equation}
H_{X Y}=\sum_{-(N-1) / 2}^{(N-1) / 2} \tilde{\Lambda}_{k} \tilde{d}_{k}^{\dagger} \tilde{d}_{k}
\end{equation}
with 
\begin{equation}
\tilde{\Lambda}_{k}:=\sqrt{\left(\lambda-\cos \phi_{k}\right)^{2}+\left(\gamma \sin \phi_{k}\right)^{2}}
\end{equation}
where the latter expression allow us to identify the critical regions of the model.
\subsection{Fermionic Covariance Matrix for the XY model}

As we mentioned before, sometimes it turns out to be better, and useful to work directly with the Covariance matrix. To be able to do so,we need to express the Hamiltonian \eqref{CH3:Hamiltonian_XY} in terms of Majoranana fermions. This can be done by using an analogous of the Jordan Wigner transformation use to diagonalised the XY Hamiltonian but now we apply it to the $2N$ Majorana fermions

\begin{equation}
\hat{\gamma}_{l}=\left(\prod_{m<l} \hat{\sigma}_{m}^{z}\right) \hat{\sigma}_{l}^{x}, \quad \hat{\gamma}_{l+N}=\left(\prod_{m<l} \hat{\sigma}_{m}^{z}\right) \hat{\sigma}_{l}^{y},
\end{equation}
where again $l=1,2\ldots N-1$
\begin{figure}[H]
    \centering
    \includegraphics[width=0.5\textwidth]{Figures/ecuacion.png}
    \caption{Illustration of how the spins in the chain are mapped to the Majorana fermions.}
    \label{majorana fermions}
\end{figure}
and similarly as before we have that
\begin{equation}
\hat{\gamma}_{l} \hat{\gamma}_{l+N}=\left(\prod_{m<l} \hat{\sigma}_{m}^{z}\right)\left(\prod_{m<l} \hat{\sigma}_{m}^{z}\right) \hat{\sigma}_{l}^{x} \hat{\sigma}_{l}^{y}=i \hat{\sigma}_{l}^{z},
\end{equation}
\begin{equation}
\hat{\gamma}_{l+N} \hat{\gamma}_{l+1}=\left(\prod_{m<l} \hat{\sigma}_{m}^{z}\right) \hat{\sigma}_{l}^{y}\left(\prod_{m<l+1} \hat{\sigma}_{m}^{z}\right) \hat{\sigma}_{l+1}^{x}=\hat{\sigma}_{l}^{y} \hat{\sigma}_{l}^{z} \hat{\sigma}_{l+1}^{x}=i \hat{\sigma}_{l}^{x} \hat{\sigma}_{l+1}^{x},
\end{equation}
\begin{equation}
\hat{\gamma}_{l} \hat{\gamma}_{l+N+1}=\left(\prod_{m<l} \hat{\sigma}_{m}^{z}\right) \hat{\sigma}_{l}^{x}\left(\prod_{m<l+1} \hat{\sigma}_{m}^{z}\right) \hat{\sigma}_{l+1}^{y}=\hat{\sigma}_{l}^{x} \hat{\sigma}_{l}^{z} \hat{\sigma}_{l+1}^{y}=-i \hat{\sigma}_{l}^{y} \hat{\sigma}_{l+1}^{y}.
\end{equation}
Which coincide, up to constant factors, with the three terms in the Hamiltonian \eqref{CH3:Hamiltonian_XY}. This will lead us to a Hamiltonian of the form\cite{botero_bcs-like_2004,latorre_ground_2004}
\begin{equation}
H_{X Y}=\frac{i}{4} \sum_{\alpha, \beta=0}^{2 N} \Omega_{\alpha \beta}\left[\hat{\gamma}_{\alpha}, \hat{\gamma}_{\beta}\right],
\label{CH3:Hamiltonian_to_diagonalise}
\end{equation}
where $\Omega$ is the antisymmetric matrix of the form
\begin{equation}
\Omega=\left[\begin{array}{c|c}
0 & \tilde{\Omega} \\
\hline -\tilde{\Omega}^{T} & 0
\end{array}\right],
\label{CH3:Block_matrix}
\end{equation}
with 
\begin{equation}
\tilde{\Omega}=\begin{pmatrix}
\lambda & \frac{1-\gamma}{2} & 0 &0 &\ldots  &0 &\frac{1+\gamma}{2}\\
\frac{1+\gamma}{2} & \lambda & \frac{1-\gamma}{2} & 0 &\ldots &0 &0\\
0 & \frac{1+\gamma}{2} & \lambda & \frac{1-\gamma}{2} &\ldots &0 &0\\
\vdots& \ddots & \ddots & \ddots & \ldots &  \vdots & \vdots\\
\frac{1-\gamma}{2}&0&0&0&\ldots & \frac{1+\gamma}{2} & \lambda.
\end{pmatrix}.
\label{CH3:Hamiltonian_matrix_XY_model}
\end{equation}
In general \eqref{CH3:Block_matrix} can be diagonalised via an orthogonal transformation $O$\cite{botero_bcs-like_2004,latorre_ground_2004} \footnote{This special relation provide us a way to transform from spacial modes to excitation in the chain, so that we can either excite the chain and see what the spatial modes are or the other way.}
\begin{equation}
    \Omega=O\left[\begin{array}{c|c}
0 & \omega \\
\hline \omega & 0
\end{array}\right] O^T,
\label{CH3:Matrix_decomposed}
\end{equation}
where $O\in O(2N)$.  Writing it in terms of two smaller orthogonal matrices will look as
\begin{equation}
O=\left[\begin{array}{c|c}
O_1 & 0 \\
\hline 0 & O_2
\end{array}\right],
\end{equation}
and $\omega$ is a diagonal matrix of size $N\times N$ which holds excitation numbers $-1/2+n$. By doing the product of matrices in \eqref{CH3:Matrix_decomposed} we can easily see that 
\begin{equation}
    \tilde{\Omega}=O_1 \omega O_2^T,
\end{equation}
which is nothing but the singular value decomposition of the matrix $\tilde{\Omega}$. The latter result tell us that a fast way to construct the matrix $O$, which diagonalise $\Omega$, is to focus on $\tilde{\Omega}$.
\newline
A fact that we can exploit is that , the matrix described in equation \eqref{CH3:Hamiltonian_matrix_XY_model} $\tilde{\Omega}$ is a circulant real matrix, meaning that it can be easily diagonalised by means of a Fourier transform. So we can write
\begin{equation}
\tilde{\Omega}_{m n}=\frac{1}{N} \sum_{\theta_{k} \in(-\pi, \pi)} \omega\left(\theta_{k}\right) e^{\phi\left(\theta_{k}\right)} e^{i(m-n) \theta_{k}}
\label{CH3:circulant_expantion}
\end{equation}
where $\omega\left(\theta_{k}\right)=\omega\left(\theta_{k}\right)^{*}=\omega\left(-\theta_{k}\right), \phi\left(\theta_{k}\right)=-\phi\left(\theta_{k}\right)$ and are given by
\begin{equation}
\omega^{2}\left(\theta_{k}\right):=\left(\lambda-\cos \theta_{k}\right)^{2}+\gamma^{2} \sin ^{2} \theta_{k},
\end{equation}
and
\begin{equation}
\phi\left(\theta_{k}\right):=\arctan \left(\frac{\lambda-\cos \theta_{k}}{-\gamma \sin \theta_{k}}\right).
\end{equation}
So expanding the equation \eqref{CH3:circulant_expantion}, we get
\begin{equation}
\begin{aligned}
\tilde{\Omega}_{m n} &=\frac{1}{N}\left[\omega(0)+(-1)^{m-n} \omega(\pi)+2 \sum_{0<\theta_{k}<\pi} \omega\left(\theta_{k}\right) \cos \left(\theta_{k}(m-n)+\phi\left(\theta_{k}\right)\right)\right] \\
&=\frac{\omega(0)}{N}+(-1)^{m-n} \frac{\omega(\pi)}{N}+\sum_{0<\theta_{k} \leq \pi} \omega\left(\theta_{k}\right)\left(u_{m}^{c}\left(\theta_{k}\right) v_{n}^{c}\left(\theta_{k}\right)+u_{m}^{s}\left(\theta_{k}\right) v_{n}^{s}\left(\theta_{k}\right)\right),
\end{aligned}
\end{equation}

where
\begin{equation}
u_{m}^{c}\left(\theta_{k}\right)=\sqrt{\frac{2}{N}} \cos \left(m \theta_{k}+\phi\left(\theta_{k}\right)\right), \quad u_{m}^{s}\left(\theta_{k}\right)=\sqrt{\frac{2}{N}} \sin \left(m \theta_{k}+\phi\left(\theta_{k}\right)\right),
\end{equation}
\begin{equation}
v_{n}^{c}\left(\theta_{k}\right)=\sqrt{\frac{2}{N}} \cos \left(n \theta_{k}\right), \quad u_{n}^{s}\left(\theta_{k}\right)=\sqrt{\frac{2}{N}} \sin \left(n \theta_{k}\right).
\end{equation}
Now defining $
u^{s}(0)=v^{s}(\pi)=0 \mathrm{y} u^{c}(0)=v^{c}(\pi)=\frac{1}{\sqrt{N}}$, we have that $\tilde{\Omega}_{m,n}$ can be written as
\begin{equation}
\tilde{\Omega}_{m n}=\sum \omega\left(\theta_{k}\right)\left(u_{m}^{c}\left(\theta_{k}\right) v_{n}^{c}\left(\theta_{k}\right)+u_{m}^{s}\left(\theta_{k}\right) v_{n}^{s}\left(\theta_{k}\right)\right).
\end{equation}
Therefore the upper part of the Hamiltonian reads
\begin{equation}
H=\sum_{m, n=0}^{N-1} \frac{i}{4} \sum_{\theta_{k}=0}^{\pi} \omega\left(\theta_{k}\right)\left(u_{m}^{c}\left(\theta_{k}\right) v_{n}^{c}\left(\theta_{k}\right)+u_{m}^{s}\left(\theta_{k}\right) v_{n}^{s}\left(\theta_{k}\right)\right)\left[\hat{\gamma}_{n}, \hat{\gamma}_{m+N}\right],
\end{equation}

\begin{equation}
H=\sum_{\theta_{k}=0}^{\pi} \omega\left(\theta_{k}\right)(\underbrace{\left[\hat{\gamma}_{k}^{c}, \hat{\gamma}_{k+N}^{c}\right]}_{1-2\sigma^{z     }_k}+\underbrace{\left[\hat{\gamma}_{k}^{s}, \hat{\gamma}_{k+N}^{s}\right]}_{1-2\sigma^{z}_k}),
\end{equation}

where
\begin{equation}
\hat{\gamma}_{k}^{c, s}:=\sum_{n} u_{n}^{c, s}\left(\theta_{k}\right) \hat{\gamma}_{n}, \quad \hat{\gamma}_{k+N}^{c, s}:=\sum_{n} v_{n}^{c, s}\left(\theta_{k}\right) \hat{\gamma}_{n+N}.
\end{equation}
Now we look back on the fact that the Fermionic covariance matrix, defined by $
\Gamma_{\alpha \beta}=\frac{1}{2 i} \operatorname{tr}\left(\rho\left[\gamma_{\alpha}, \gamma_{\beta}\right]\right)=\frac{1}{2 i}\langle \left[\gamma_{\alpha}, \gamma_{\beta}\right] \rangle$, that brings $\Omega$ into its Williamson form, does the same on the Fermionic covariance matrix. Thus for a state $|\vec{n}\rangle$, consider an eigenstate of the base $(c,s,\theta_k)$, where $m^{c, s}\left(\theta_{k}\right)-1 / 2$, with $n^{c, s}\left(\theta_{k}\right)$ the occupation number of \textit{cosine}, \textit{sine} in the $k-$mode. We get 

\begin{equation}
\begin{aligned}
\tilde{\Gamma}_{m n} &=\sum_{\theta_{k}}^{\pi}\left[m^{c}\left(\theta_{k}\right) u_{m}^{c}\left(\theta_{k}\right) v_{n}^{c}\left(\theta_{k}\right)+m^{s}\left(\theta_{k}\right) u_{m}^{s}\left(\theta_{k}\right) v_{n}^{s}\left(\theta_{k}\right)\right] \\
&=\sum_{\theta_{k}}^{\pi}\left(\frac{m^{c}\left(\theta_{k}\right)+m^{s}\left(\theta_{k}\right)}{2}\right)\left(u_{m}^{c}\left(\theta_{k}\right) v_{n}^{c}\left(\theta_{k}\right)+u_{m}^{s}\left(\theta_{k}\right) v_{n}^{s}\left(\theta_{k}\right)\right) \\
&+\sum_{\theta_{k}}^{\pi}\left(\frac{m^{c}\left(\theta_{k}\right)-m^{s}\left(\theta_{k}\right)}{2}\right)\left(u_{m}^{c}\left(\theta_{k}\right) v_{n}^{c}\left(\theta_{k}\right)-u_{m}^{s}\left(\theta_{k}\right) v_{n}^{s}\left(\theta_{k}\right)\right).
\end{aligned}
\end{equation}
by defining $m^{\pm}\left(\theta_{k}\right)=\frac{m^{c}\left(\theta_{k}\right) \pm m^{s}\left(\theta_{k}\right)}{2}$ and inverting the transformations done above, we finally get that
\begin{equation}
\tilde{\Gamma}_{m n}=\overbrace{\sum_{\theta_{k}}^{\pi} m^{+}\left(\theta_{k}\right) e^{i \phi\left(\theta_{k}\right)} e^{i(n-m) \theta_{k}}}^{\tilde{\Gamma}^{+}_{mn}}+\underbrace{\sum_{\theta_{k}}^{\pi} m^{-}\left(\theta_{k}\right) e^{i \phi\left(\theta_{k}\right)} e^{i(n+m) \theta_{k}}}_{\tilde{\Gamma}_{m n}^{-}}.
\end{equation}
We notice that $\tilde{\Gamma}^{+}_{mn}$ is circulant, whereas $\tilde{\Gamma}^{-}_{mn}$ is not, nevertheless, observe that $\tilde{\Gamma}^{+}_{mn} = \tilde{\Gamma}^{}_{mn'}$, with $n'$ a change on the index $n\to -n'$, which can be interpreted as a rotation over the circle.
\begin{figure}[H]
    \centering
    \includegraphics[width=0.6\textwidth]{Figures/Reflection_over_circle.png}
    \caption{Meaning of the relabel done in the circulant matrix, which can be seen as a reflection over the circle.}
    \label{reflectioncircle}
\end{figure}
Explicitly we can write that if $\tilde{\Gamma}^{+}_{mn}$ has the shape
\begin{equation}
\left(\begin{array}{ccccc}
a_{0} & a_{-1} & \cdots & a_{2} & a_{1} \\
a_{1} & a_{0} & \cdots & a_{3} & a_{2} \\
\cdot & \cdot & \cdot & \cdots & \cdot \\
\vdots & \vdots & \vdots & \vdots & \vdots \\
a_{-1} & a_{-2} & a_{-3} & \cdots & a_{0}
\end{array}\right),
\end{equation}
then $\tilde{\Gamma}^{-}_{mn}$ will be given by
\begin{equation}
\left(\begin{array}{ccccc}
a_{0} & a_{1} & \cdots & a_{-2} & a_{-1} \\
a_{1} & a_{2} & \cdots & a_{-1} & a_{0} \\
\cdot & \cdot & \cdot & \cdots & \cdot \\
\vdots & \vdots & \vdots & \vdots & \vdots \\
a_{-1} & a_{0} & a_{1} & \cdots & a_{-2}
\end{array}\right).
\end{equation}

So we can spot 3 things. First, the FCM always can be written as a circulant matrix plus an anticirculant matrix. Second, in the Ground state, the FCM is circulant only, since the fermion occupation numbers $n^{c}\left(\theta_{k}\right)=n^{s}\left(\theta_{k}\right)=0, \forall k$. Third, for a generic state, we have that in average the FCM matrix is always circulant, because $\langle n^{c}\left(\theta_{k}\right)\rangle=\langle n^{s}\left(\theta_{k}\right)\rangle$.\\
Now that we have showed a full characterization of the $XY$ model, we change gears and start to board our problem. In the next section we are going to show how this is possible to find bound for the case of the $XY$ model and we will be able to show that the mechanism for Ultra Orthogonality in this system, is related with the structure of the system.


\section{Exploring ``Ultra Orthogonality''}
Now having all the tools we need we can apply what we discussed in section $2$. First we will look at the error exponents for the $XY$ model, we will compute the probabilities of having errors and show some numerical results. Then we will move to our next problem, we will show for the $XY$ model is possible to bound the determinant and even more we find an analytical result for this determinant based on a generalisation of Szeg\''{o} limit theorems. 

\subsection{Error exponent}

First we recall the fact that in order to sample states at a given temperature we make use of a very well known technique, named Gibbs sampling. This technique 




