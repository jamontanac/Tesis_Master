\chapter*{Introduction}

Alternative considerations about the foundations of statistical mechanics proposed by Posescu et. al.\cite{popescu_foundations_2005},  have shown that reliability on subjective randomness, ensemble averaging or time averaging are not required, instead, a quantum information perspective provides an answer, from an uniquely quantum point of view. This way of tackling the problem consider the universe\footnote{Here we refer to the Universe as the system together with a sufficiently large environment.} in a quantum pure state. Due to entanglement between system and environment, thermalisation appears as a generic property of the pure states of the universe subjected to a global constraint. This result is mathematically proven and is known as ``General Canonical Principle'', specifically, what this principle tell us is that every system will be thermalised for almost all pure states of the universe. In other words, every Hilbert subspace, small enough compared to its environment, will drive to the same state, the canonical state. Even though this principle provides a general viewpoint of thermalisation, this result is kinematic, rather than dynamical. That is, particular unitary evolution of the global state is not considered, and thermalisation is not proven to happen, instead, the key ingredient is Levy's Lemma \cite{milman_asymptotic_2009,noauthor_concentration_nodate}, which plays a similar role to the law of large numbers and governs the properties of typical states in large- dimensional Hilbert spaces\cite{popescu_foundations_2005}, and provide a powerful tool to evaluate functions of randomly chosen quantum states\footnote{It is important to stress the fact that these ideas were not only proposed by Popescu et. al., contenporarly,  Gremmer. et al\cite{gemmer_quantum_2004} proposed something similar, nonetheless, the viewpoint we will dicuss here is mostly based on Popescu's work.}. However, because almost all states of the universe bring the system to its canonical state, one could anticipate that most evolutions will quickly carry a state in which the system is not thermalised, to the one in which it is, and that the system will remain thermalised for most of its evolution.\\


\cite{horodecki_partial_2005}
The present of this work arises as a result of a research in the termalisation problem.  

empezamos por la parte de mecanica estadistica pero no es bueno centrarlo en eso

Originalmente se queria mirar el proceso de termalización pero no se hizo.

La termalizacion da una luz a cual es el mecanismo, pues de alguna manera.


La historia es que si tengo un subespacio de hilbert en donde parece que todos los estados conducen al mismo estado, el canonico, al tomar dos estados psi 1 y psi 2


