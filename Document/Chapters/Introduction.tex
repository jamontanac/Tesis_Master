
\chapter*{Introduction}

Alternative considerations about the foundations of statistical mechanics proposed by Posescu et. al.\cite{popescu_foundations_2005},  have shown that reliability on subjective randomness, ensemble averaging or time averaging are not required, instead, a quantum information perspective\cite{horodecki_partial_2005} provides an answer, from an uniquely quantum point of view. This way of tackling the problem consider the universe\footnote{Here we refer to the Universe as the system together with a sufficiently large environment.} in a quantum pure state. Due to entanglement between system and environment, thermalisation appears as a generic property of the pure states of the universe subjected to a global constraint. This result is mathematically proven and is known as ``General Canonical Principle'', specifically, what this principle tell us is that every system will be thermalised for almost all pure states of the universe. In other words, every Hilbert subspace, small enough compared to its environment, will drive to the same state, the canonical state. Even though this principle provides a general viewpoint of thermalisation, this result is kinematic, rather than dynamical. That is, particular unitary evolution of the global state is not considered, and thermalisation is not proven to happen, instead, the key ingredient is Levy's Lemma \cite{milman_asymptotic_2009,ledoux_concentration_2005}, which plays a similar role to the law of large numbers and governs the properties of typical states in large- dimensional Hilbert spaces\cite{popescu_foundations_2005}, and provide a powerful tool to evaluate functions of randomly chosen quantum states\footnote{It is important to stress the fact that these ideas were not only proposed by Popescu et. al., contenporarly,  Gremmer. et al \cite{gemmer_quantum_2004}, as well as Goldstein et. al \cite{goldstein_canonical_2006}, proposed similar ideas, nonetheless, the viewpoint we will discuss here is mostly based on Popescu's work.}.  Additionally, since all states of the universe bring the system to its canonical state, one could anticipate that most evolutions will quickly carry a state in which the system is not thermalised, to the one in which it is, and that the system will remain thermalised for most of its evolution.\\\\
\indent The latter problem motivated the research conducted by Linden et. al.\cite{linden_quantum_2009}, in which by using arguments based on typicality's ideas, reaching equilibrium can be proven to be an universal property of quantum systems. In the frame of these ideas, dynamical aspects are address to explore the evolution which drives systems to equilibrate, and moreover, to study under which circumstances equilibrium is reached on systems. The series of results shown in \cite{linden_quantum_2009,linden_speed_2010,malabarba_quantum_2014}, suggest that first, virtually any subsystem of any large enough system will reach equilibrium and fluctuate around it at almost all times, and second, that average speed fluctuations are extremely small at almost all times, meaning that the state of the system remains relatively static around its equilibrium. \\\\\
\indent It turns out that typicality, is an alternative way of studying thermalisation, as well as, equilibration on quantum systems. Even though these techniques are an extremely useful when understanding the foundations of statistical mechanics, there is an implication from typicality which haven not been studied in depth\footnote{Of course, there is a possibility these ideas have been pointed out by someone else before, nonetheless, it seems it has not been of great interest for others since there is not a publication nor a public discussion about this subject that we could find.}, which is the core idea of this work, and what we consider could be another alternative to understand equilibration in systems. To illustrate what we mean, consider two different pure states living on the same Hilbert subspace ($\ket{\psi_1},\ket{\psi_2} \in \mathcal{H}_{R}$), the one that is obtained by imposing a global constraint over the universe. From typicality we know that the reduce state of $\ket{\psi_1}$ and $\ket{\psi_2}$ leads to the same state, the canonical state\footnote{Explicitly what this means is that
\begin{equation*}
\operatorname{Tr}_{\mathcal{E}}\ket{\psi_1}\bra{\psi_1}=\operatorname{Tr}_{\mathcal{E}}\ket{\psi_2}\bra{\psi_2}=\Omega_{\mathcal{S}},
\end{equation*}
where $\Omega_{\mathcal{S}}$ corresponds to the canonical state of the system.}.
%Therefore, motivated by this usefulness, here we address an implication of typicality, which we state could be another alternative to understand equilibration process and even more thermalisation in systems. Namely, from  typicality, we know that two different states of the universe subject to the same global constraint, will both lead to the same  reduced state, the canonical state
Thus, from this fact, we could consider a third state $\ket{\psi_3}$ which is a generic linear combination of $\ket{\psi_1}$ and $\ket{\psi_2}$, if the condition of typicality is also imposed over the third state, we have therefore that its reduced state will also lead us to the canonical state, meaning that cross terms obtained in the reduced state associated with $\ket{\psi_3}$ should somehow vanish. Explicitly, when we compute the density matrix associated with the state $\ket{\psi_3}$ we have 
\begin{equation}
\ket{\psi_3}\bra{\psi_3} = |c_1|^2 \ket{\psi_1}\bra{\psi_1} + |c_2|^2 \ket{\psi_2}\bra{\psi_2}+c_1^*c_2\ket{\psi_2}\bra{\psi_1} + c_2^*c_1\ket{\psi_1}\bra{\psi_2},
\end{equation}
and then its reduced state reads
\begin{equation}
\operatorname{Tr}_{\mathcal{E}}\ket{\psi_3}\bra{\psi_3} \equiv \Omega_{\mathcal{S}}= \Omega_{\mathcal{S}}+c_1^*c_2\operatorname{Tr}_{\mathcal{E}}\ket{\psi_2}\bra{\psi_1} + c_2^*c_1\operatorname{Tr}_{\mathcal{E}}\ket{\psi_1}\bra{\psi_2},
\label{Intro:equation_super_orthogonality}
\end{equation}
where the condition of normalisation was used ($|c_1|^2 + |c_2|^2 =1$). Notice that the cross terms in equation \eqref{Intro:equation_super_orthogonality} should vanish in order to satisfy the equality, namely, the condition which has to be satisfied in order to keep the equality is $\operatorname{Tr}_{\mathcal{E}}\ket{\psi_2}\bra{\psi_1}$ $=$ $\operatorname{Tr}_{\mathcal{E}}\ket{\psi_1}\bra{\psi_2}$ $=0$.  We name this condition ``Super Orthogonality'', since when we compute the partial trace over the exterior product of $\ket{\psi_1}$ and $\ket{\psi_2}$ it becomes zero.\\\\
\indent The last-mentioned property, may seem a little useless, but indeed this very property is related with the dynamics of the system. To illustrate this, consider an state which is time dependent $\ket{\Psi(t)}$, we can expand this state in its energy eigenstates as
\begin{equation}
\ket{\Psi(t)}=\sum_{k} c_k e^{-iE_kt}\ket{E_k},
\end{equation}
where $\sum_k|c_k|^2=1$, hence
\begin{equation}
\rho(t) = \sum_{ k,\ell}c_kc^*_\ell e^{-i(E_k-E_\ell)t}\ket{E_k}\bra{E_\ell} = \underbrace{\sum_{ k}|c_k|^2 \ket{E_k}\bra{E_k}}_{\omega}+\sum_{ k,\ell}c_kc^*_\ell e^{-i(E_k-E_\ell)t}\ket{E_k}\bra{E_\ell},
\end{equation}
so when we look at the reduced state of $\rho(t)$ will lead to
\begin{equation}
\operatorname{Tr}_{\mathcal{E}} \rho(t) \equiv \rho_{\mathcal{S}}(t) =\omega_{\mathcal{S}} + \sum_{ k,\ell}c_kc^*_\ell e^{-i(E_k-E_\ell)t}\operatorname{Tr}_{\mathcal{E}}\ket{E_k}\bra{E_\ell},
\end{equation}
where $\omega_{\mathcal{S}} = \operatorname{Tr}_{\mathcal{E}} \omega$. Thus if instead of taking arbitrary states in equation \eqref{Intro:equation_super_orthogonality}, we could take states on the energy eigenbasis we could have that for certain situations the equilibrium could be reach immediately whenever super orthogonality holds. Having this in mind, we stress again the motivation of studying super orthogonality as an alternative way of reaching equilibrium.\\\\
\indent Indeed as we just discussed, the motivation behind super orthogonality is related with the equilibration and moreover since for most cases we expect thermal typicality to be present in the system, the equilibrium state will coincide with the thermal one. Even though to full study this property could provide us an alternative picture of the equilibration as an instantaneous process, in the present work we will not provide either an idea of how this might be an alternative mechanism,  nor a connection between equilibration and thermalisation, in fact, in this work we tackle a problem inspired by super orthogonality which tries to describe the describe full dynamical states of the universe such that when we compute its reduced state, it will immediately be stationary.



