\begin{center}
\begin{figure}
\centering%
\epsfig{file=HojaTitulo/LogoAndes.PNG,scale=.2}%
\end{figure}
\thispagestyle{empty} \vspace*{0.0cm} \textbf{\huge
 Exploring equilibration in fermionic systems: \\ A connection with minimum distance codes} \\[4.0cm]
\Large\textbf{Jose Alejandro Montaña Cortes}\\[5.5cm]
\small Universidad de los Andes\\
Faculty of Science, Department of Physics\\
Bogotá, Colombia\\
2020\\
\end{center}

\newpage{\pagestyle{empty}\cleardoublepage}

\newpage
\begin{center}
\thispagestyle{empty} \vspace*{0cm} \textbf{\huge Exploring equilibration in fermionic systems: \\ A connection with minimum distance codes } \\[1.8cm]
\Large\textbf{Jose Alejandro Montaña Cortes}\\[2.5cm]
\textbf{Supervisor:}\\[1.0cm]
Alonso Botero Mejía\\[2.0cm]
\small A thesis submitted in fulfilment of the requirements for the degree of:\\
\textbf{Master's Degree in Physics}\\
In the research group of:\\
Quantum Mechanics and Information Theory\\[2.0cm]
Universidad De Los Andes\\
Faculty of Science , Departament of Physics\\
Bogotá, Colombia\\
2020\\


%L\'{\i}nea de Investigaci\'{o}n:\\
%Nombrar la l\'{\i}nea de investigaci\'{o}n en la que enmarca la tesis  o trabajo de investigaci\'{o}n\\

\end{center}

%\newpage{\pagestyle{empty}\cleardoublepage}

\newpage{}
%\thispagestyle{empty} \textbf{}\normalsize
%\\\\\\%
%\textbf{(Dedicatoria o un lema)}
\quad\\[5.0cm]

\begin{flushright}
\begin{minipage}{8cm}
    \noindent
        \small
 %       Su uso es opcional y cada autor podr\'{a} determinar la distribuci\'{o}n del texto en la p\'{a}gina, se sugiere esta presentaci\'{o}n. En ella el autor dedica su trabajo en forma especial a personas y/o entidades.\\[1.0cm]\\
 %       Por ejemplo:\\[1.0cm]
       % ``\textit{To my family, to my friends and specially to Valentina}''\\
        .....
 %       o\\[1.0cm]
 %       La preocupaci\'{o}n por el hombre y su destino siempre debe ser el
  %      inter\'{e}s primordial de todo esfuerzo t\'{e}cnico. Nunca olvides esto
   %     entre tus diagramas y ecuaciones.\\\\
    %    Albert Einstein\\
\end{minipage}
\end{flushright}


%\newpage{\pagestyle{empty}\cleardoublepage}
%\newpage{}

%\thispagestyle{empty} \textbf{}\normalsize
%\\\\\\%
%\textbf{\LARGE Acknowledgments}
%\addcontentsline{toc}{chapter}{\numberline{}Acknowledgments}\\\\
%Esta secci\'{o}n es opcional, en ella el autor agradece a las personas o instituciones que colaboraron en la realizaci\'{o}n de la tesis  o trabajo de investigaci\'{o}n. Si se incluye esta secci\'{o}n, deben aparecer los nombres completos, los cargos y su aporte al documento.\\

%\newpage{\pagestyle{empty}\cleardoublepage}
\newpage{\pagestyle{empty}\cleardoublepage}
\newpage{}


%\textbf{\LARGE Abstract}
\addcontentsline{toc}{chapter}{\numberline{}Resumen}



%\textbf{\small Keywords: }\\[2.0cm]
\textbf{\LARGE Resumen}\\\\
\
La tipicidad canónica ha surgido como una alternativa a los fundamentos de la mecánica estadística, logrando explicar desde una perspectiva de la información cuántica, fenómenos como el de la termalización, que emerge como consecuencia del entrelazamiento entre el sistema y su ambiente.  Posteriores trabajos han mostrado que esta particular forma de abordar el problema proporciona una idea del mecanismo por el cual se alcanza el equilibrio en sistemas cuánticos. Dentro del marco de estas ideas, promedios temporales juegan un rol principal a la hora de entender cómo estados reducidos obtenidos a partir de un estado puro, alcanzan el equilibrio. Esto nos conduce a preguntarnos: ¿Es posible encontrar estados reducidos provenientes de un estado puro no estacionario, que automáticamente estén equilibrados, es decir, estados reducidos cuyo equilibrio se alcance de forma inmediata?, y en caso de ser así, ¿es posible determinar el tamaño del conjunto de estados cumpliendo esta propiedad?\\
\indent En el presente trabajo, el uso de herramientas de teoría de código, específicamente, códigos aleatorios fermiónicos de distancia mínima, proporciona una forma diferente de entender los sistemas fermiónicos reducidos. Confirmando así, la  existencia de estados reducidos que permanecen en equilibrio para estos sistemas. En este sentido, se explora el espacio de Hilbert generado por el conjunto de estos estados, concretamente, se caracteriza el espacio de Hilbert asociado a estos estados por medio de un exponente de error, el cual proporciona una ley de grandes desviaciones en el sistema. De esta forma, se prueba que el tamaño del espacio de Hilbert asociado a los estados, que cumplen la propiedad de equilibrarse de forma inmediata, es exponencialmente grande.\\\\
\textbf{\small Palabras Clave:} Tipicidad, Sistemas Fermiónicos, códigos aleatorios fermiónicos de distancia mínima .
\newpage{}
\addcontentsline{toc}{chapter}{\numberline{}Abstract}
\textbf{\LARGE Abstract}\\\\
\
Canonical typicality has emerged as an alternative to the foundations of statistical mechanics. It has been able to explain from a viewpoint of quantum information theory, how thermalisation emerge from entanglement between the system and its environment. Later results have shown that the ideas used in typicality, provide an explanation to the mechanism of the evolution towards equilibrium for large quantum systems. Within this framework of ideas, time averages plays a major rol when it comes to understand how reduced states, obtained from a pure states, can reach equilibrium. This led us to ask ourselves, is it possible to find reduced states obtained from a non stationary pure state, such that they equilibrate instantly, that is, reduced states such that immediately reach its equilibrium state? If so, is it possible to measure the size of the space fulfilling this property? \\
\indent In the present work, tools of code theory, specifically fermionic minimum distance codes, provide an alternative understanding of fermionic systems. Specifically, we characterised the size of the Hilbert space generated by the states, such that fulfil this property, throughout an error exponent, which provides a characterisation of the large deviation law within the system. Hence, we prove that the Hilbert space of states fulfilling the property of reaching its equilibrium instantly is indeed exponentially large.\\\\
\textbf{\small Key Words: }Typicality, fermionic systems, fermionic minimum distance codes.
%\newpage{\pagestyle{empty}\cleardoublepage}
\newpage{}

%\newpage{\pagestyle{empty}\cleardoublepage}
\addcontentsline{toc}{chapter}{\numberline{}Agradecimientos}
\textbf{\LARGE Agradecimientos}
%\\\\
%Esta secci\'{o}n es opcional, en ella el autor agradece a las personas o instituciones que colaboraron en la realizaci\'{o}n de la tesis  o trabajo de investigaci\'{o}n. Si se incluye esta secci\'{o}n, deben aparecer los nombres completos, los cargos y su aporte al documento.\\

%\newpage{\pagestyle{empty}\cleardoublepage}
\newpage{}
\addcontentsline{toc}{chapter}{\numberline{}Declaration of Authorship}
\textbf{\LARGE Declaration of Authorship}\\\\
\
I, Jose Alejandro Montaña Cortes, declare that this thesis titled,`` \textit{Exploring equilibration in Fermionic systems: A connection with minimum distance codes}'' and the work presented in it are my own. I confirm that:

\begin{itemize} 
\item This work was done wholly or mainly while in candidature for a research degree at this University.
\item Where any part of this thesis has previously been submitted for a degree or any other qualification at this University or any other institution, this has been clearly stated.
\item Where I have consulted the published work of others, this is always clearly attributed.
\item Where I have quoted from the work of others, the source is always given. With the exception of such quotations, this thesis is entirely my own work.
\item I have acknowledged all main sources of help.
\item Where the thesis is based on work done by myself jointly with others, I have made clear exactly what was done by others and what I have contributed myself.\\
\end{itemize}
 
\noindent Signed:\\\\

\rule[0.5em]{25em}{0.5pt} % This prints a line for the signature
 
\noindent Date:\\\\

\rule[0.5em]{25em}{0.5pt} % This prints a line to write the date
\newpage{}