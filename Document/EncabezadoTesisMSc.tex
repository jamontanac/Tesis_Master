\PassOptionsToPackage{dvipsnames}{xcolor}
\documentclass[12pt,english,openany,letterpaper,pagesize]{book}%{scrbook}%fleqn
\renewcommand{\familydefault}{\rmdefault}
\usepackage[pagebackref]{hyperref}
\hypersetup{
    colorlinks,
    citecolor=black,
    filecolor=black,
    linkcolor=black,
    urlcolor=black
}

\usepackage{enumitem}
\usepackage[utf8]{inputenc}
\usepackage[english]{babel}
\usepackage{fancyhdr}
\usepackage{epsfig}
\usepackage{cite}
\usepackage{dsfont}
%\usepackage{apacite}
%\usepackage{epic}
%\usepackage{eepic}
\usepackage{epigraph}
\usepackage{amsmath}
\usepackage{xr}
\usepackage{caption}
\usepackage{subcaption}
%\usepackage{physics}
\usepackage{float}
%\usepackage{threeparttable}
%\usepackage{amscd}
%\usepackage{here}
\usepackage{graphicx}
%\usepackage{lscape}
\usepackage{tabularx}
%\usepackage{subfigure}
%\usepackage{longtable}
\usepackage{braket}
\usepackage[dvipsnames]{xcolor}
\usepackage{cancel}

%\usepackage[titles]{tocloft}

%\usepackage{rotating}
 %Para rotar texto, objetos y tablas seite. No se ve en DVI solo en PS. Seite 328 Hundebuch
                        %se usa junto con \rotate, \sidewidestable ....
\usepackage{listings}
\lstset{basicstyle=\selectfont\ttfamily ,
  showstringspaces=false, frame=single,	 
  commentstyle=\color{blue},
  keywordstyle=\color{violet},
  otherkeywords={append},
 literate=
            *{[}{{\textcolor{NavyBlue}{[}}}{1}
            {]}{{\textcolor{NavyBlue}{]}}}{1}
            {(}{{\textcolor{OliveGreen}{(}}}{1}
            {)}{{\textcolor{OliveGreen}{)}}}{1},
}
\usepackage{amsthm}
\usepackage{amsmath}
\usepackage{amsfonts}
\usepackage{amssymb}
\usepackage{tikz}
\usetikzlibrary{arrows}
\usepackage{braket}
\usepackage{empheq}
\usepackage{textpos}
\usepackage{verbatim}
\usepackage{pgfplots}
\usepackage{listings}

\usepackage{pgfplots}
\usepackage{xcolor}
\usepackage{alltt}
\usetikzlibrary{mindmap, trees}
\usetikzlibrary{shapes.geometric}
\usetikzlibrary{arrows}
\usetikzlibrary{shadings}
\usepackage{verbatim}
\usetikzlibrary{decorations.pathreplacing}
\usepackage{pgfplots}
\usepackage{amssymb}

%\renewcommand{\theequation}{\thechapter-\arabic{equation}}
%\renewcommand{\thefigure}{\textbf{\thechapter-\arabic{figure}}}
%\renewcommand{\thetable}{\textbf{\thechapter-\arabic{table}}}
\pagestyle{fancyplain}\addtolength{\headwidth}{\marginparwidth}
\textheight22.5cm \topmargin0cm \textwidth16.5cm
\oddsidemargin0.5cm \evensidemargin-0.5cm%
\renewcommand{\chaptermark}[1]{\markboth{\thechapter\; #1}{}}
\renewcommand{\sectionmark}[1]{\markright{\thesection\; #1}}
\lhead[\fancyplain{}{\thepage}]{\fancyplain{}{\rightmark}}
\rhead[\fancyplain{}{\leftmark}]{\fancyplain{}{\thepage}}
\fancyfoot{}
\thispagestyle{fancy}%

\addtolength{\headwidth}{0cm}
\unitlength1mm %Define la unidad LE para Figuras
%\mathindent0cm %Define la distancia de las formulas al texto,  fleqn las descentra
\marginparwidth0cm
\parindent0.5cm %Define la distancia de la primera linea de un parrafo a la margen
%Para tablas, redefine el backschlash en tablas donde se define la posici\'{o}n del texto en las
%casillas (con \centering \raggedright o \raggedleft)
\newcommand{\PreserveBackslash}[1]{\let\temp=\\#1\let\\=\temp}
\let\PBS=\PreserveBackslash
%Espacio entre lineas
\renewcommand{\baselinestretch}{1.3}
%Neuer Befehl f\"{u}r die Tabelle Eigenschaften der Aktivkohlen
\newcommand{\arr}[1]{\raisebox{1.5ex}[0cm][0cm]{#1}}
\DeclareMathOperator{\Tr}{Tr}
%Neue Kommandos
%\usepackage{Befehle}
\usepackage{braket}

\newtheorem{theorem}{Theorem}[section]
\newtheorem{definition}{Definition}[section]

