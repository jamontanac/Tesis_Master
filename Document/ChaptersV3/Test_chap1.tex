Boltzmann vs Gibbs entropy

la distincion entre Boltzmann y Gibbs es escencialmente que al considerar un sistema de muchas particulas tal como un gas diluido, el estado de cada particula se le puede describir por un punto en el espacio de Fase de 1 particula. (particulas puntuales -> espacio de 6 dimensiones).
Cuando se tiene un gas, este se puede describir como un histograma en el espacio de fase de 1 particula. Reticulamos el espacio de fase y miramos cuantos puntos hay en cada celda. de esta forma podemos hablar de un histograma que llamamos F(q,p) que nos indica cuantas particulas hay de forma dinamica en el estado q,p en el instante t. El histograma es una funcion del estado microscopico, por ende se puede construir un programa de computador proporcionandole unas condiciones iniciales dadas y por ende el histograma será entonces una función determinista de estas condiciones iniciales. F no es un probabilidida, es un numero de ocupacion instantaneo que nos dice cuantas particulas hay con este estado particular. F es desconocido pues no se conocen las condiciones iniciales pero en el momento en que las tenga, un gas particular, el que tengo en mi mesa, ese F es una cantidad que puede que yo la ignore pues pueda ser que yo no lo pueda calcular pero tiene un Valor determinado para Dios. No es algo aleatorio, está determinado. De igual forma H de Boltzmann es una función que es determinista, Es como la energia de un sistema de N particulas en el que conozco las posiciones iniciales y momentos iniciales, entonces yo ya sé cuanto vale eso. El argumento de boltzmann es que al mirar este tipo de funciones como H, pensando en el espacio de fase de $N$ particulas. tomamos una region del espacio de fase que sea tipica. por ende el espacio de fase 6N dimensional, a cada punto lo coloreamos con el valor del histograma, entonces de que color será el espacio de Fase. La idea de Boltmann es que el espacio de fase, se ve del color histograma donde el H es maximo. practicamente todos los puntos del espacio de fase tienen la propiedad de que cuando miro el histograma de l espacio de fase de 1 particula, el histograma que más se presenta sobre todo el espacio de fase, de la gran mayoria, es el histograma que maximiza H, histograma canonico. Acá no hay probabilididad. el valor más comun que toma una funcion sobre un espacio. La probabilidad viene de las condiciones iniciales. Tome un cilindro partido en 2 y remueve ekl tabique de la izquierda. y vamos a seguir el histograma. lo que se ve es que el histograma va a asumir el histograma canonico de 1 particula. Si viera todas las condiciones iniciales microscopicaas que son compatibles con la condicion inicial macroscopica. nos damos cuenta que estas condiciones nos llevan a una region del espacio de fase donde el valor más frecuente del histograma sera el canonico.Mandar todas las particulas de forma separada horizontalmente con la misma velocidad, entonces nunca colisionan y el histograma es una cosa completamente atipica pero esto es muy esepcional. Lanzamiento de monedas, en 1000 lanzamientos tenemos 2^1000 resultados, si cada uno fuera una trayectoriua independiente, y vemos una funcion estadistica (caras y sellos) entonces el historgama mas comun es el histograma 1/2 1/2, y en casi 2^1000 casos tengran ese histograma, y la proporcion de que un histograma difiera por mas de sqrt(N) es exponencialemente pequeño. Boltzmann no tiene una funcion de distribucion de N particulas y no se habla en terminos de distribuciones conjuntas. F es un numero de coupacion de estados de 1 particula.


Boltmann tiene problemas en el caso en que no tenemos gases diluidos, y todo el trabajo de boltzmann es intentar incorporar interacciones para lograr extender sus ideas a una funcion que describa ocupacion de pares y define las integrales de colision y todo en aras de lograr entenedr como funcionan las cosas en sitemas interactuantes y cuando es fuertemente interactuantes, nosotros pensamos como Gibbs , F es como el valor esperado de una funcion de ocupacion, por teoremas de intercambiabilidad en distribuciones de probabilidad, esos valores esperados coinciden con las probabilidades marginales.
En un sistema interactuante una distribucion canonica marginal desde el punto de vista de boltmann como podemos definir la energia, como se puede separar el termino de interaccion. y para el caso de gibbs si se puede pero como podemos cuadrar esto con la distribucion de 1 particula, entonces esto termina siendo un intento por cuadrar algo que no tiene mucha forma. El problema importante es que Boltzmann no funciona para comparar con las entropias empiricas de experimentos y la idea de la Mec est es explicar la termodinamica.



Por otro lado Gibbs, trabaja con distribuciones de probbilidad sobre el sistema de  N particulas, miramos un macroestado que define una region en el espacio de fase de N particulas. Define una distribucion de probabilidad sobre el estado de fase de N particulas y sobre esto saco las preducciones y la entropia se calcula a partir de la distribucion de N particulas no como la de boltmann que es de 1 particula. El caso de el gas diluido se tiene entonces que H de Boltzmann mas comun nos da la entropia de Gibbs multiplicada por N. por que la F mas frecuente coincide con en ese caso con el valor esperado ( mas frequente) y el valor esperado de F es la probabilidad marginal de una gas diluido, la conjunta es el producto de las marginales y por ende sólo en ese caso. Gibbs es realmente la aproximacion de teoria de informacion esencial. Yo no séª. tenemos una medida de probabilidad.

La hipotesis de probabilidad apriori es una idea muy bien pensada y discutimos sobre los fundamentos de la mecanica estadistica 

Iguales probabilidades a priori quiere decir que hay una medida canonica en el espacio de fase, admite un sistema de cordenadas por el teorema de darboux de los espacios simplecticos y hay una forma canonica que es la forma simplectica. Iguales volumenes tienen igual probabilidad y sólo cuando se mide con la medida simplectica, no con cualquier otra medida, si el espacio de fase lo hicieramos con posiciones y velocidades, entonces al medir una medida uniforme en esas coordenadas no tendríamos las mismas predicciones de la mecanica estadistica. y es por que esta medida es invariante ante transformaciones canonicas. La approx de gibbs es asignar probabilidades sobre el espacio de fase de N particulas, y esta distribucion evoluciona canonicamente con el Hamiltoniano del sistema de las N particulas y si no, el sistema se debe entender como una parte de un sistema más grande pero de igual forma es necesario asignar probabilidades sobre el sistema mas grande, evolucionar el sistema más grande usando evolucion Hamiltoniana y marginalizar. 


Gibbs en fisica estadistica cuantica, se traduce a decir que si solo sé que el sitema tiene una energia dada, entonces todos los estados cuanticos puros que estén en ese subespacio.
en un sistema cerrado, la maxima identificacion del sistema es un estado puro. un estado mezclado es una especificacion incompleta por ignorancia o por que el sistema es parte de un sistema más grande. El estado de liouville es un estado de informacion de la persona o del observador, pero el estado del sistema es una coleecion de posicion y momentos que puede ser desconocida.
El sistema esta en un microestado desconocido, entonces asignamos probabilidades con una medida de probabilidad, la descripcion del sistema en cuantica es un estado puro, y este vive en un subespacio del espacio de Hilbert, y asignamos probabilidades a los estados puros, y la medida designada que es invariante ante transformaciones unitarias es la que que se hereda de la medida de Haar, espacio de Hilbert proyectivo, la medida que sale de la metrica de Fubini study. Si tratamos de justificas la asignacion gibbsiana, pensemos en un estado canonica ocmo un estado basado en la informacion disponible sobre el sistema. Haciendo las distinciones que son relevantes del sistema y del subsistema

Desde gibbs podemos argumentar que el macrocanonico es un 

La forma de ver nuestras cosas es Boltzmaniana con relacion a este problema, pues Boltmzmann lo que nos dice es que para el caso del Gas, uno toma una condicion inicial cualquiera, entonces eso sigue una trayectoria y demas, entonces despues de un tiempo tengo una collecion de mommentos y posicionesexactos y computamos el histograma y entonces lo que passa es que todo el tiempo se tiene que estamos cerca a los valores del histograma de ocupaciones de una particula que corresponde a la distribucion canonica. De alguna forma para mabas aproximaciones la probabilidad está ahi pero se usa de forma distinta, Boltzmann dice que quedese toda la vida evolucione y vea como se ve el histograma, nos podemos quedar toda la vida viendo el histograma y vea que esto cambia dinamicamente y tendremos entonces el mismo histograma todo el tiempo, existe una direccionalidad. Gibbs si promediamos sobre todas las trayectorias, entonces el histograma promedio corresponde con este histograma que habla boltmann, Pero boltmann dice que en practicamente todas las instancias individuales esto va a suceder.


Promedio de ensamble es distinto al promedio hablado con boltmzmann, 

Veo lo qeu hacemos como una aproximacion como neoBoltmaniana, lo que quiere decir botlzmann es que el proyector de gibbs que sale de la mecanica cuantica, esto se puede pensar como una integral sobre todos los estados de ese subespacio, el subespacio restringido, entonces cualquier valor esperado que yo saque con ese proyector o sus matrices de densidad reducidas puede ser entendido como un promedio sobre los estados puros individuales o sus matrices de densidades individuales. si tengo el proyector y miro la canonica, entonces puedo ver que la canonica es exactamtne igual como la integral sobre el subespacio restringido de las matrices de densidad reducidas rho_psi, es el promedio de estas matrices. Al marginalizar, predecimos unas cnatindades termodinamics, esto puede ser entendido como el promedio de los psi de los valores esperados de estas cantidades. Pero lo que decimos es que casi todos los terminos en esa integral contribuyen de igual forma. Esto no está incorporado apriori en gibbs. El poder de ver las cosas asi es que. Mire el espacio de condiciones iniciales y va a ver que casi todas las condiciones nos llevan a una funcion f que maximiza la funcion H.

El dificil manejar un lenguaje tan preciso que contenga todas estas sutilezas




rotular los modos las excitaciones las regiones, relevancia con los modos como se conecta con el problma mas macro