
\chapter*{Introduction}

Alternative considerations about the foundations of statistical mechanics proposed by Posescu et. al.\cite{popescu_foundations_2005},  have shown that reliability on subjective randomness\cite{ma_statistical_1985}, ensemble averaging\cite{farquhar_ergodic_1965} or time averaging\cite{jancel_foundations_2013} are not required to understand the emergence of thermalisation; instead, a quantum information perspective\cite{horodecki_partial_2005} provides an answer to it, from a uniquely quantum point of view, which does not rely on any ignorance probabilities in the description of the state. This precise way of tackling the problem is intimately related to empiricism. Put a bottle of beer in the refrigerator, so we expect it will be cold after some time. It may not, because electricity in the house may fail, or the refrigerator may be damaged,  or any other reason, but we know that it is not something normal to happen, so we know that the ``typical'' thing to happen is that the beer will cool down after some time. Similarly, what Popescu et. al. proved is that if we consider a quantum pure state, subject to a global constraint, the ``typical'' thing to happen is that the reduced state of the system is the canonical mixed state. That is, not only the state of the system mixed, but it is in precisely the state we would expect from standard statistical arguments \cite{popescu_foundations_2005}, which is achieved by considering  subjective randomness \cite{deutsch_thermodynamic_2010,singh_foundations_2013}. To provide a more precise argument, consider the universe (i.e the system together with a sufficiently large environment) in a quantum pure state. Due to entanglement between system and environment, system thermalisation appears as a generic property of pure states of the universe subjected to a global constraint, as well as the high dimensional spaces associated with the states. This result is mathematically proven\cite{popescu_entanglement_2006} and is considered to be an important result when understanding statistical mechanics. This result is known as``General Canonical Principle'', or informally as ``Typicality'' . Specifically, what this principle tells us is that whenever we look at a sufficiently small system, compared to its environment, the reduced state of the system will approximately correspond to the thermal state \cite{popescu_foundations_2005,popescu_entanglement_2006,goldstein_canonical_2006,gemmer_quantum_2004}, and therefore showing that thermalisation occurs as a generic property of pure states of the universe. Even though this principle provides a general viewpoint of thermalisation, this result is kinematic, rather than dynamical. That is, particular unitary evolution of the global state is not considered at any moment, and thermalisation is not proven to happen; instead, the key ingredient is Levy's Lemma \cite{milman_asymptotic_2009,ledoux_concentration_2005}, which plays a similar role to the law of large numbers and governs the properties of typical states in large-dimensional Hilbert spaces\cite{popescu_foundations_2005}, and thus provides a powerful tool to evaluate functions of randomly chosen quantum states. These ideas were not only proposed by Popescu et. al., contemporary to them,  Gremmer. et al \cite{gemmer_quantum_2004}, as well as Goldstein et. al \cite{goldstein_canonical_2006}, proposed similar ideas, nonetheless, the viewpoint we discuss here is mostly based on the one proposed by Popescu.\\
 

   
\indent Since the vast overwhelming majority of states in the universe bring the system to the canonical mixed state, which is proven to be the expected from standard statistical arguments, one could anticipate that most evolutions will quickly carry a state in which the system is not thermalised to the one in which it is, and that the system will remain thermalised for most of its evolution. Motivated by this, Linden et. al.\cite{linden_quantum_2009} conducted a research to answer if thermalisation could happen as a universal property of quantum systems. Thus, by using arguments based on ideas of typicality, reaching equilibrium can be proven to be typical property of large quantum systems. In the frame of these ideas, dynamical aspects are addressed to explore the evolution that drives systems to equilibrate, and moreover, to study under what circumstances systems reach equilibrium and how much they fluctuate. The series of results shown in \cite{linden_quantum_2009,linden_speed_2010,malabarba_quantum_2014}, suggest that virtually any subsystem of a sufficiently large system will reach equilibrium and fluctuate around it at almost all times. Where the only conditions required are , that the Hamiltonian has no degenerate energy gaps, which is equivalent to say that the energy levels are non-degenerate\footnote{To be precise, this tell us that if any non-zero difference of eigenenergies determines the two energy values involved; i.e, for any four eigenstates with energies $E_n$,$E_m$, $E_k$ and $E_\ell$, $E_k- E_\ell = E_m - E_n$ implies $k=\ell$ and $m=n$ or $k=m$ and $n=\ell$.}, and that the state of the universe contains sufficiently many energy eigenstates. These conditions are fulfilled for most of physical situations. First, all but a measure zero of Hamiltonians have non-degenerate energy gaps, and the main reason is that whenever an infinitesimally small random perturbation enters in the system, degeneracies will be removed. Second, the vast majority of states in the Hilbert space are such that they contain very many eigenstates.\\



\indent Even though thermalisation seems to be a very straight forward process, a closer look, reveals that thermalisation is composed by many different aspects, where equilibration, bath state independence, subsystem state independence and the Boltzmann form of the equilibrium state play the main role\cite{linden_quantum_2009}. First, equilibration is understood as the process in which the system reaches a particular state and remains in that state or close or to it for almost all times. Whenever we refer to equilibration any structure of the equilibrium is inferred; Indeed, this state may depend on initial conditions. Second, bath state independence refers to the fact that the equilibrium state of the system should not depend on the precise initial state of the Bath, meaning that only macroscopic parameters are needed to described the bath. Third, subsystem state independence is related to the fact that the equilibrium reached by the system should be independent of its initial state. And finally forth, the Boltzmann form of the equilibrium state describes the Boltzmanninan form of the equilibirum state ($\rho_{\mathcal{S}}=\frac{1}{Z}\operatorname{exp}(-\frac{H_{\mathcal{S}}}{k_B T})$). Therefore, typicality turns out to be a useful and an alternative way of studying thermalisation, as well as, equilibration on quantum systems. Even though these techniques are extremely useful when understanding the foundations of statistical mechanics, there is an implication from typicality which may be strongly connected with Eigenstate Thermalisation Hypothesis (ETH)\cite{srednicki_chaos_1994,deutsch_quantum_1991,rigol_alternatives_2012}, and what we consider worth the effort to study thoroughly. Being this the reason and the core idea of this work, and what we consider could yield to further insights to understand equilibration in systems. To illustrate our ideas, consider two different orthogonal pure states living on the same Hilbert subspace ($\ket{\psi_1},\ket{\psi_2} \in \mathcal{H}_{R}$), the one that is obtained by imposing a global constraint over the universe. From typicality we know that the reduce state of $\ket{\psi_1}$ and $\ket{\psi_2}$ approximately leads to the same state, that is, 
\begin{equation}
\operatorname{Tr}_{\mathcal{E}}\ket{\psi_1}\bra{\psi_1}\approx\operatorname{Tr}_{\mathcal{E}}\ket{\psi_2}\bra{\psi_2}\approx\Omega_{\mathcal{S}},
\end{equation}
where $\Omega_{\mathcal{S}}$ corresponds to the canonical state of the system.
%Therefore, motivated by this usefulness, here we address an implication of typicality, which we state could be another alternative to understand equilibration process and even more thermalisation in systems. Namely, from  typicality, we know that two different states of the universe subject to the same global constraint, will both lead to the same  reduced state, the canonical state
Thus, from this fact, we could consider a third state $\ket{\psi_3}$ which is a generic linear combination of $\ket{\psi_1}$ and $\ket{\psi_2}$, if the condition of typicality is also imposed over the third state. We have therefore that its reduced state will also lead us to the canonical state, meaning that cross terms obtained in the reduced state associated with $\ket{\psi_3}$ should somehow vanish. Explicitly, when we compute the density matrix associated with the state $\ket{\psi_3}$ we have 
\begin{equation}
\ket{\psi_3}\bra{\psi_3} = |c_1|^2 \ket{\psi_1}\bra{\psi_1} + |c_2|^2 \ket{\psi_2}\bra{\psi_2}+c_1^*c_2\ket{\psi_2}\bra{\psi_1} + c_2^*c_1\ket{\psi_1}\bra{\psi_2},
\end{equation}
and then its reduced state reads
\begin{equation}
\operatorname{Tr}_{\mathcal{E}}\ket{\psi_3}\bra{\psi_3} \equiv \Omega_{\mathcal{S}}= \Omega_{\mathcal{S}}+c_1^*c_2\operatorname{Tr}_{\mathcal{E}}\ket{\psi_2}\bra{\psi_1} + c_2^*c_1\operatorname{Tr}_{\mathcal{E}}\ket{\psi_1}\bra{\psi_2},
\label{Intro:equation_super_orthogonality}
\end{equation}
where the condition of normalisation was used ($|c_1|^2 + |c_2|^2 =1$). Notice that the cross terms in equation \eqref{Intro:equation_super_orthogonality} should vanish in order to satisfy the equality, namely, the condition which has to be satisfied in order to keep the equality is $\operatorname{Tr}_{\mathcal{E}}\ket{\psi_2}\bra{\psi_1}$ $=$ $\operatorname{Tr}_{\mathcal{E}}\ket{\psi_1}\bra{\psi_2}$ $=0$. As one might think, this could be connected to ETH, since off diagonal terms vanish, except that for this case we can not clearly see if the off diagonal terms go to zero as $|E_i-E_j|$ becomes larger. Therefore, to put this condition in a common and colloquial language, we name this condition ``Ultra Orthogonality''. This particular name was given since when we compute the partial trace over the exterior product of $\ket{\psi_1}$ and $\ket{\psi_2}$ it becomes zero, thus providing an idea of Orthogonality under partial traces.\\

\indent The last-mentioned property may seem a little confusing and hard to fulfil, but indeed this very property can be proven to be related with the dynamics of the system. To illustrate this, consider a state which is time dependent $\ket{\Psi(t)}$, we can expand this state in its energy eigenstates as
\begin{equation}
\ket{\Psi(t)}=\sum_{k} c_k e^{-iE_kt}\ket{E_k},
\end{equation}
where $\sum_k|c_k|^2=1$, hence
\begin{equation}
\rho(t) = \sum_{ k,\ell}c_kc^*_\ell e^{-i(E_k-E_\ell)t}\ket{E_k}\bra{E_\ell} = \underbrace{\sum_{ k}|c_k|^2 \ket{E_k}\bra{E_k}}_{\omega}+\sum_{ k,\ell}c_kc^*_\ell e^{-i(E_k-E_\ell)t}\ket{E_k}\bra{E_\ell},
\end{equation}
so when we look at the reduced state of $\rho(t)$ will lead to
\begin{equation}
\operatorname{Tr}_{\mathcal{E}} \rho(t) \equiv \rho_{\mathcal{S}}(t) =\omega_{\mathcal{S}} + \sum_{ k,\ell}c_kc^*_\ell e^{-i(E_k-E_\ell)t}\operatorname{Tr}_{\mathcal{E}}\ket{E_k}\bra{E_\ell},
\label{Intro:equation_of_evolution}
\end{equation}
where $\omega_{\mathcal{S}} = \operatorname{Tr}_{\mathcal{E}} \omega$. Thus, if instead of taking arbitrary states in equation \eqref{Intro:equation_super_orthogonality}, we could take states on the energy eigenbasis such that for certain situations equilibrium could be reached immediately whenever Ultra Orthogonality holds. Having this in mind, we stress again the motivation of studying Ultra Orthogonality as an alternative way of reaching equilibrium.\\

\indent The motivation behind Ultra Orthogonality is quite related with the equilibration, as well as with thermalisation\footnote{ Since for most cases we expect thermal typicality to be present in the system, the equilibrium state will coincide with the thermal one.}. Of course, it would be quite interesting to full study this property to provide a better comprehension of what is behind Ultra Orthogonality, and provide what may appear as an alternative mechanism for equilibration. Nonetheless, for this work we are not going to talk much about the general problem, instead we will be discussing a particular case of Ultra Orthogonality. Here we are going to describe the reduced state of a pure and full dynamical states of the universe, such that its reduced states, sufficiently small compared the environment, are automatically stationary. Namely, we will be discussing the case when the cross terms in \eqref{Intro:equation_of_evolution} are exactly equal to zero ($\operatorname{Tr}_{\mathcal{E}}\ket{E_k}\bra{E_\ell} = 0$) in the special case of systems that can be mapped into a Fermionic-like system via some non local transformation such as Jordan Wigner transformation.\\

\indent One of the first questions one might wonder is, ``are there few or conversely , many states that fulfil the property that the cross terms in \eqref{Intro:equation_of_evolution} are exactly equal to zero, and if so, what is the size of the Hilbert subspace associated with these states?''. We will provide the answers to these questions by using some ideas taken from random minimum codes adapted to Fermionic systems, and we will show through a rate exponent that there are exponentially large Hilbert subspaces in which all its states fulfil the condition $\operatorname{Tr}_{\mathcal{E}}\ket{E_k}\bra{E_\ell} = 0$.\\

\indent This document is divided in four chapters: In the first chapter, we are going to introduce canonical typicality and its relation to statistical mechanics, starting with the postulate of equal a priori probability and the controversy around this argument. Next we pass through typicality and finally with a detailed explanation of the problem of Ultra Orthogonality. In the second chapter, we will state general definitions concerning Fermionic states, in particular, we will discuss the formalism of Grassmann algebras and Grassmann Variables. In the third chapter we are going to provide some definitions with regard to correction codes and specifically some of the well known bounds for minimum distance codes, more over, in this chapter we present our main result which we exhibit as follows, first, an explicit connection between the Fermionic systems and random minimum distance codes is formulated in order to understand Ultra Orthogonality for Fermionic systems, and second, we provide an estimation of the size of the Hilbert subspace associated with the states that fulfil Ultra Orthogonality. Finally, in the fourth chapter we present our conclusions as well as our perspective of our work.


