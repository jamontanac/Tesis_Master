\documentclass[12pt]{article}
\usepackage[utf8]{inputenc}

\title{Exploring Thermalisation From The Viewpoint Of Typicality In The XY Model}
%Explorando la termalización desde el puntode vista de la tipicidad en el modelo XY
\author{Jose Alejandro Montana Cortes}

\begin{document}
\maketitle

\section*{Abstract}
The foundation of the statistical mechanics is still a subject of continuous controversy. Being one of the most controversial problems, the validity of the equal a priori probability postulate, which can not be proved. However, recent ideas from \textit{Popescu et. al}\cite{popescu_entanglement_2006} leave behind this postulate, replacing it with the idea of \textit{Canonical Typicality}, which can be mathematically proved. In the frame of these ideas, entanglement between the system and its environment is the main key to prove that the vast majority of reduced states from an universe subject to a global condition, approximately coincide with the equiprobable state of this universe.
\newline
From Typicality an interesting, but non trivial, property emerges when considering pure states, such that are a linear combination of eigenstates of the Hamiltonian. When considering this kind of states, one can see that by applying a partial trace over these states, cross terms must vanish in order for typicality to hold. We call this property ``Ultra-orthogonality'' under partial traces.
\newline
In this work we show an study of this very property for the case of the $XY$ 1-dimensional model, i which the ``Gaussianity'' of the states can be exploited to efficiently compute the state of the whole chain as well as its reduced states (portions of the chain), and then give us a picture of what is behind the mechanism of this property which emerges from general principles.

\bibliography{references.bib} 
\bibliographystyle{ieeetr}

\end{document}