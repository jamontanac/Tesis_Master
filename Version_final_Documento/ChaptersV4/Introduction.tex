
\chapter*{Introduction}

In the last two decades, a wave of works in quantum information theory has proven to be a new and objective form of understanding the foundations of statistical mechanics \cite{popescu_foundations_2005, goldstein_canonical_2006, gemmer_thermalization_2006, popescu_entanglement_2006, goldstein_approach_2010, kaufman_quantum_2016, gogolin_equilibration_2016 }. Particularly, alternative considerations about the foundations of statistical mechanics proposed by Posescu et. al.\cite{popescu_foundations_2005},  have shown that reliability on subjective randomness\cite{ma_statistical_1985}, ensemble averaging\cite{farquhar_ergodic_1965} or time averaging\cite{jancel_foundations_2013} are not required to understand the emergence of thermalisation; instead, a quantum information perspective\cite{horodecki_partial_2005} provides an alternative answer to the foundations of statistical mechanics, from a purely quantum point of view, which does not rely on any ignorance probabilities in the description of the state. This precise way of tackling the problem is intimately related to empiricism, for example: Whenever you plan your day,  or you go out for a walk, the last thing that goes through our mind is to get smacked by a meteor, we know it may happen, but we know that it is not something ``normal'' to occur. That is why it is ``typical'' to plan our lives without even bothering by getting hit by a meteor. Similarly, what Popescu et. al. \cite{popescu_foundations_2005} proved is that if we consider a quantum pure state, subject to a global constraint (e.g a constant energy), the ``typical'' thing to happen is that the reduced state of the system is very close to the canonical mixed state. That is, almost every reduced state obtained from a quantum pure state will approximately coincide with the canonical thermodynamic state \cite{deutsch_thermodynamic_2010, singh_foundations_2013}.\\

\indent To provide a more precise argument, consider as the universe ($\mathcal{U}$) the system ($\mathcal{S}$) together with a sufficiently large environment ($\mathcal{E}$), in a quantum pure state. Due to entanglement between the system and the environment and properties of high dimensional Hilbert spaces, system thermalisation appears as a local generic property of pure states of the universe subjected to a global constraint. This result is known as the \textit{general canonical principle}, or informally as \textit{canonical typicality} and is considered to be an important result when understanding statistical mechanics in quantum systems. Specifically, what this principle tells us is that whenever we look at a sufficiently small system, compared to its environment, the reduced state of the system will approximately correspond to the thermal state \cite{popescu_foundations_2005, popescu_entanglement_2006, goldstein_canonical_2006, gemmer_quantum_2004},  therefore suggesting that thermalisation occurs as a generic local property of pure states of the universe.\\
\indent It should be emphasised that the results in typicality apart from providing a general viewpoint of thermalisation, the nature of those arguments is kinematic, rather than dynamic. That is, the particular unitary evolution of the global state is never considered, and thermalisation is not proven to happen; instead, the key ingredient is Levy's Lemma \cite{milman_asymptotic_2009, ledoux_concentration_2005}, which plays a similar role to the law of large numbers and governs the properties of typical states in large-dimensional Hilbert spaces\cite{popescu_foundations_2005}, and thus provides a powerful tool to evaluate functions of randomly chosen quantum states. We stress here that these ideas were not only proposed by Popescu et. al. \cite{popescu_foundations_2005}; contemporaneously with them,  Gemmer. et. al. \cite{gemmer_quantum_2004}, as well as Goldstein et. al. \cite{goldstein_canonical_2006}, proposed similar ideas, in which heuristic arguments are used to prove canonical typicality, and exhibit an explicit connection between reduced states an the micro-canonical density matrix at a suitable total energy $E$. However, the result obtained by Popescu et. al. establishes canonical typicality in a general way by invoking the Levy's Lemma\cite{popescu_foundations_2005,milman_asymptotic_2009,ledoux_concentration_2005},  For that reason, the viewpoint we discuss here is mostly based on the one proposed by Popescu.\\   
\indent With the purpose of extending typicality beyond the kinematic viewpoint and address the dynamics of thermalisation, we enquire under what conditions the state of the universe will evolve into the typical region of its Hilbert Space in which its subsystems are thermalised and remain in that space for most of its evolution. Motivated by previous results heading this direction\cite{tasaki_quantum_1998,gemmer_thermalization_2006} and the fact that, from typicality arguments is possible to show that the overwhelming majority of states in the universe bring the system to the canonical mixed state, Linden et. al.\cite{linden_quantum_2009} explore whether thermalisation could happen as a universal property of quantum systems. Thus, by using arguments based on ideas of typicality, reaching equilibrium can be shown to be a typical property of large quantum systems. In this framework, dynamical aspects are addressed to explore the evolution that drives systems to equilibrate, moreover, to study under what circumstances systems reach equilibrium and how much they fluctuate about the equilibrium state. A series of results in \cite{linden_quantum_2009, linden_speed_2010, malabarba_quantum_2014} suggest that under mild conditions, any subsystem of a sufficiently large system will reach equilibrium and fluctuate around it at almost every time. The only conditions required are that the Hamiltonian has no degenerate energy gaps, and that the state of the universe contains sufficiently many energy eigenstates. These conditions are fulfilled for most physical situations, in fact all but a set of measure zero of Hamiltonians have non-degenerate energy gaps\cite{linden_quantum_2009}.\\
% and the main reason is that whenever an infinitesimally small random perturbation enters in the system, degeneracies will be removed. Second, the vast majority of states in the Hilbert space are such that they contain very many eigenstates.\\
%\footnote{To be precise, this tell us that if any non-zero difference of eigenenergies determines the two energy values involved; i.e, for any four eigenstates with energies $E_n$,$E_m$, $E_k$ and $E_\ell$, $E_k- E_\ell = E_m - E_n$ implies $k=\ell$ and $m=n$ or $k=m$ and $n=\ell$.}
\indent Even though thermalisation seems to be a very straightforward process, it is quite difficult to formalise an explanation to it. A closer look, reveals that thermalisation is composed of many different aspects that have to be inspected in detail, and where \textit{equilibration}, \textit{bath state independence}, \textit{subsystem state independence} and \textit{the Boltzmann form of the equilibrium state} play a prominent role\cite{linden_quantum_2009}. First, \textit{equilibration} is the process in which the system reaches a particular state and remains in that state or close to it. Whenever we refer to equilibration, note that any particular state is not inferred and in general it does not need to be a thermal state. Second, \textit{bath state independence} refers to the fact that the equilibrium state of the system should not depend on the precise initial state of the Bath. That is, only macroscopic parameters are needed to describe the bath\cite{linden_quantum_2009}, for example, its temperature: In the moment equilibrium is reached, that state should only depend on the temperature of the bath. Third, \textit{subsystem state independence} refers to the fact that the equilibrium state reached by the system, should be independent of its initial state. Finally, the \textit{Boltzmann form of the equilibrium state} refers to the form of the equilibrium state $\rho_{\mathcal{S}}=\frac{1}{Z}\operatorname{exp}(-\frac{H_{\mathcal{S}}}{k_B T})$, which is known as a Boltzmanninan form of equilibrium. Note that equilibration is then a more general process which can depend on different parameters such as initial conditions in an arbitrary way, whereas thermalisation does not.\\
%\indent Before going any further, it is important to deconstruct the fundamentals of thermalisation  and understand what is needed to be proven in a particular system to assure that the system thermalises\cite{linden_quantum_2009}. 
%\begin{itemize}
%\item \textbf{Equilibration:} It is understood as the process in which the system reaches a particular state and remains in that state or close or to it for almost all times. Whenever we refer to equilibration, note that any particular is not inferred and in general it does not need to be a thermal state. Equilibration is a more general process that can depend on different parameters such as initial conditions in an arbitrary way, whereas thermalisation does not.
%\item \textbf{Bath state independence:} This refers to the fact that the equilibrium state of the system should not depend on the precise initial state of the Bath. That is, only macroscopic parameter are needed to describe the bath. For example, its temperature; in the moment equilibrium is reached, that state of equilibrium should only depend on the temperature of the Bath.
%\item\textbf{Subsystem state independence:} It tell us that the equilibrium state reached by the system, should be independent of its initial state. 
%Undoubtedly, nature has shown us that the thermal state of equilibrium is obtained independent of its parameters.
%\item \textbf{Boltzmann form of the equilibrium:} Under certain additional conditions of the Hamiltonian, the equilibrium state of the subsystem can be written as $\rho_{\mathcal{S}}=\frac{1}{Z}\operatorname{exp}(-\frac{H_{\mathcal{S}}}{k_B T})$.
%\end{itemize}
\indent Realising that thermalisation is compound by the afore-mentioned elements, let us clarify an important distinction between thermalisation and equilibration, indeed, we will consider equilibration as a general quantum phenomenon that may occur in situations other than those associated with thermalisation. By using this decomposition of thermalisation, Linden et. al. were able to prove the first two elements mentioned above (Equilibration and bath state independence). Namely, they prove not only that reaching equilibrium is a universal property of quantum systems but that this equilibrium state does not depend on the precise details of the bath state, but rather on its macroscopic parameters \cite{linden_quantum_2009}.\\
 %Therefore, typicality turns out to be a useful and an alternative way of studying thermalisation, as well as, equilibration on quantum systems. Even though these techniques are extremely useful when understanding the foundations of statistical mechanics, there is an implication from typicality which may be strongly connected with Eigenstate Thermalisation Hypothesis (ETH)\cite{srednicki_chaos_1994,deutsch_quantum_1991,rigol_alternatives_2012}, and what we consider worth the effort to study thoroughly. Being this the reason and the core idea of this work, and what we consider could yield to further insights to understand equilibration in systems.

\indent Up to this point, we have shown that typicality has been proven to be an extremely useful alternative way of studying thermalisation in quantum systems and understanding the foundations of quantum statistical mechanics. Nonetheless, a closer look to typicality will derive in a property that we consider could yield to further insights to understand equilibration in quantum systems. To illustrate our ideas, consider two different orthogonal pure states, $\ket{\psi_1}$ and $\ket{\psi_2}$, living in the same Hilbert subspace $\mathcal{H}_{R}$ ($\ket{\psi_1},\ket{\psi_2} \in \mathcal{H}_{R}$), the one that is obtained by imposing a global constraint, denoted by $R$, over the universe. From typicality we know that the reduced state of $\ket{\psi_1}$ and $\ket{\psi_2}$ approximately leads to the same state, that is, 
\begin{equation}
\operatorname{Tr}_{\mathcal{E}}\ket{\psi_1}\bra{\psi_1}\approx\operatorname{Tr}_{\mathcal{E}}\ket{\psi_2}\bra{\psi_2}\approx\Omega_{\mathcal{S}},
\end{equation}
where $\Omega_{\mathcal{S}}$ corresponds to the canonical state of the system.
%Therefore, motivated by this usefulness, here we address an implication of typicality, which we state could be another alternative to understand equilibration process and even more thermalisation in systems. Namely, from  typicality, we know that two different states of the universe subject to the same global constraint, will both lead to the same  reduced state, the canonical state
Thus, we could consider a third state $\ket{\psi_3} = c_1\ket{\psi_1} + c_2\ket{\psi_2} $ which is a generic linear combination of $\ket{\psi_1}$ and $\ket{\psi_2}$. We have then that its reduced state will also lead us to the canonical state, meaning that cross terms obtained in the reduced state associated with $\ket{\psi_3}$ should somehow vanish. Explicitly, when we compute the density matrix associated with the state $\ket{\psi_3}$ we have 
\begin{equation}
\ket{\psi_3}\bra{\psi_3} = |c_1|^2 \ket{\psi_1}\bra{\psi_1} + |c_2|^2 \ket{\psi_2}\bra{\psi_2}+c_1^*c_2\ket{\psi_2}\bra{\psi_1} + c_2^*c_1\ket{\psi_1}\bra{\psi_2},
\end{equation}
and hence its reduced state reads
\begin{equation}
\operatorname{Tr}_{\mathcal{E}}\ket{\psi_3}\bra{\psi_3} \approx \Omega_{\mathcal{S}}\approx \Omega_{\mathcal{S}}+c_1^*c_2\operatorname{Tr}_{\mathcal{E}}\ket{\psi_2}\bra{\psi_1} + c_2^*c_1\operatorname{Tr}_{\mathcal{E}}\ket{\psi_1}\bra{\psi_2},
\label{Intro:equation_super_orthogonality}
\end{equation}
where the condition of normalisation was used ($|c_1|^2 + |c_2|^2 =1$). Notice that the cross terms in equation \eqref{Intro:equation_super_orthogonality} should therefore approximately vanish in order to satisfy the relation. Namely, the condition which has to be satisfied in order to keep the equality is $\operatorname{Tr}_{\mathcal{E}}\ket{\psi_2}\bra{\psi_1}$ $=$ $\operatorname{Tr}_{\mathcal{E}}\ket{\psi_1}\bra{\psi_2}$ $=0$. Since this condition tell us that off-diagonal terms approximately vanish this brought to our mind some previous ideas proposed by Srednicki et. al.\cite{srednicki_chaos_1994,deutsch_quantum_1991,rigol_alternatives_2012} in eigenstate thermalisation hypothesis (ETH), where the off-diagonal terms are expected to be stochastic quantities with mean zero and an amplitude that is exponentially small on the number of degrees of freedom of the system. That is, the off-diagonal terms are expected to be near zero for a system with a large number of degrees of freedom. For our case, we have something similar but instead of having that the off-diagonal terms are near to zero, we have that after taking the partial trace, those terms become approximately zero. Although we are aware that the condition $\operatorname{Tr}_{\mathcal{E}}\ket{\psi_2}\bra{\psi_1}$ $\approx$ $\operatorname{Tr}_{\mathcal{E}}\ket{\psi_1}\bra{\psi_2}$ $\approx 0$ might be related with ETH in some way, due to the impossibility of making an explicit connection between them, we name it \textit{ultra-orthogonality},
\begin{equation}
\operatorname{Tr}_{\mathcal{E}}\ket{\psi_i}\bra{\psi_j} \approx 0,\quad \quad i\neq j.
\label{Intro:Condition_ultraorthogonality}
\end{equation}
This particular name was given since when we compute the partial trace over the exterior product of $\ket{\psi_i}$ and $\ket{\psi_j}$ it becomes zero, thus we consider this name provides the idea of having orthogonality over partial traces.\\
% As one might think, this could be connected to ETH, since off diagonal terms vanish, except that for this case we can not clearly see if the off diagonal terms go to zero as $|E_i-E_j|$ becomes larger. Therefore, to put this condition in a common and colloquial language, we name this condition ``Ultra Orthogonality''. This particular name was given since when we compute the partial trace over the exterior product of $\ket{\psi_1}$ and $\ket{\psi_2}$ it becomes zero, thus providing an idea of Orthogonality under partial traces.\\

\indent Ultra-orthogonality, can be shown to be related with the equilibration of systems. Consider a time dependent state $\ket{\Psi(t)}$, we can expand this state onto its energy eigenstates as
\begin{equation}
\ket{\Psi(t)}=\sum_{k} c_k e^{-iE_kt}\ket{E_k},
\end{equation}
where $\sum_k|c_k|^2=1$; hence,
\begin{equation}
\rho(t) = \sum_{ k,\ell}c_kc^*_\ell e^{-i(E_k-E_\ell)t}\ket{E_k}\bra{E_\ell} = \underbrace{\sum_{ k}|c_k|^2 \ket{E_k}\bra{E_k}}_{\omega}+\sum_{ k\neq\ell}c_kc^*_\ell e^{-i(E_k-E_\ell)t}\ket{E_k}\bra{E_\ell},
\end{equation}
where the fist term $\omega$ is time-independent, so when we look at the reduced state of $\rho(t)$, defined as  $ \rho_{\mathcal{S}}(t) = \operatorname{Tr}_{\mathcal{E}} \ket{\Psi(t)}\bra{\Psi(t)}$,
\begin{equation}
\rho_{\mathcal{S}}(t) =\omega_{\mathcal{S}} + \sum_{ k\neq\ell}c_kc^*_\ell e^{-i(E_k-E_\ell)t}\operatorname{Tr}_{\mathcal{E}}\ket{E_k}\bra{E_\ell},
\label{Intro:equation_of_evolution}
\end{equation}
where $\omega_{\mathcal{S}} = \operatorname{Tr}_{\mathcal{E}} \omega$.  Thus, if all states in equation \eqref{Intro:equation_of_evolution} satisfy the ultra-orthogonality property, then the reduced state would automatically be stationary, meaning that it will be time independent.\\


%Thus, if instead of taking arbitrary states in equation \eqref{Intro:equation_super_orthogonality}, we could take states on the energy eigenbasis such that for certain situations equilibrium could be reached immediately whenever Ultra Orthogonality holds. Having this in mind, we stress again the motivation of studying Ultra Orthogonality as an alternative way of reaching equilibrium.\\

\indent Remembering that equilibration is a general process in which the state remains for the almost every time (time independent), allows us to see that if the states in equation \eqref{Intro:equation_of_evolution} satisfy ultra-orthogonality, then ultra-orthogonality will be related with equilibration as an immediate phenomenon. Moreover, from typicality we know that the overwhelming majority of states are such that its reduced state approximately coincide with the thermal state, thus, one can anticipate that for most cases the equilibrium state that we will be studying will be the canonical thermal state.
%\indent Remembering that equilibration is a more general quantum phenomenon than thermalisation, we notice that the motivation behind ultra-orthogonality is quite related with equilibration as an immediate phenomenon. Since for most cases we expect thermal typicality to be present in the system, the equilibrium state will coincide with the thermal state, meaning that for most cases we will be studying thermal equilibrium.
 %\footnote{ Since for most cases we expect thermal typicality to be present in the system, the equilibrium state will coincide with the thermal one.}. 
It would therefore be interesting to fully study the property of ultra-orthogonality to provide a better comprehension of what is behind this phenomenon and provide what may appear to be an alternative instantaneous mechanism for equilibration. However, in this work we will not address the general problem of studying ultra-orthogonality in any quantum system; instead, we will be discussing a particular case of ultra-orthogonality. Here we will describe the reduced state of a pure and fully dynamical state of the universe, such that its reduced states, sufficiently small compared its environment, are automatically stationary. More specifically, we will be discussing the case when the cross terms in \eqref{Intro:equation_of_evolution} are \textit{exactly} equal to zero ($\operatorname{Tr}_{\mathcal{E}}\ket{E_k}\bra{E_\ell} = 0$) in the special case of systems that their Hamiltonians represent interacting systems that can be mapped, under appropriate approximations (or transformations), to Hamiltonians which are quadratic in fermionic operators of the form
\begin{equation}
\hat{H}=\sum_{i j} C_{i j} \hat{a}_{i}^{\dagger} \hat{a}_{j}+\sum_{i j}\left(A_{i j} \hat{a}_{i}^{\dagger}\hat{a}_{j}^{\dagger}+\mathrm{h.c.}\right),
\label{Intro:QuadraticHamiltonian}
\end{equation}
where $i,j$ run from $1$ to $N$, the number of modes in the system.\\

\indent One of the first questions one might wonder is, ``are there few or conversely, many states that fulfil the property that the cross terms in \eqref{Intro:equation_of_evolution} are exactly equal to zero?'' and consequently ``what is the size of the Hilbert subspace associated with these states?''. We will approach these questions by using some ideas taken from random minimum distance codes adapted to fermionic systems, and we will show through a concept taken from code theory, the \textit{random-coding exponent}, that there are exponentially large Hilbert subspaces in which all its states fulfil the condition $\operatorname{Tr}_{\mathcal{E}}\ket{E_k}\bra{E_\ell} = 0$.\\

\indent An insight to clarify how techniques from code theory can be used to study ultra-orthogonality starts by noting that fermionic states can be interpreted as occupation numbers $n_i$ in the different modes, where $n_i \in \{0,1\}$. Namely, we will be looking at systems of $N$ modes and we will call the subsystem the set of $L$ modes, where $N\gg L$.For this case, we have that the state of the system will be represented as a binary sequence $\ket{\vec{n}}=\ket{n_1,n_2,\ldots}$ in the Fock space. Now consider the states $\ket{\vec{n}_1}$ and $\ket{\vec{n}_2}$ that are mutually orthogonal eigenstates of the Hamiltonian \eqref{Intro:QuadraticHamiltonian}. We denote by $\vec{n}_1$ ($\vec{n}_2$) the sequence of excitations present in the correspondent state $\ket{\vec{n}_1}$ ($\ket{\vec{n}_1}$). Now we consider the error vector $\vec{e}_{12}=\vec{n}_1 +\vec{n}_2$, which its entries are zero whenever the $i-$th element of both sequences ($\vec{n}_1 ,\vec{n}_2$) is equal and one otherwise, observe that the higher the energy, the more excitations we have in the sequences, so the vector $\vec{n}_1+\vec{n}_2$ will be very likely to have more ones when we increase the energy. Denoting by $d$ the number of ones in the vector $\vec{e}_{12}$, as we will explain in detail in this document, it is possible to prove that when $d>L$,
 \begin{equation}
 \hat{X}_{12} = \operatorname{Tr}_{N-L}\ket{\vec{n}_1}\bra{\vec{n}_2} = 0,
 \end{equation}
where $ \operatorname{Tr}_{N-L}$ represents the partial trace over the $N-L$ elements. This condition of getting crossed terms equals to zero at certain distance $d$ brought to our minds the idea of minimum distance codes, codes that have the property of perfectly correct errors when the number of errors do not exceed $\lfloor(d-1)/2\rfloor$.   Hence, our idea is to look at the minimum distance code $\mathfrak{C}=\{x^{(1)},x^{(2)},\ldots, x^{(2^k)}\}$ with $2^k$ \textit{codewords}, such that their respective states $\{\ket{x^{(1)}},\ket{x^{(2)}},\ldots, \ket{x^{(2^k)}}\}$ are mutually ultra-orthogonal. We will be concern with the size of the largest minimum distance code that can be built to alternatively answer what is the size of the largest Hilbert subspace $\mathcal{H}_{\mathfrak{C}}$ spanned by these vectors ($\mathcal{H}_{\mathfrak{C}} = \operatorname{Span}\left(\{\ket{x^{(1)}},\ket{x^{(2)}},\ldots, \ket{x^{(2^k)}}\}\right)$). That is, if we look at a state $\ket{\psi}\in \mathcal{H}_{\mathfrak{C}}$, it can be expanded in terms of the basis vectors $\ket{x^{(i)}}$
\begin{equation}
\ket{\psi} = \sum_{\vec{n}\in \mathfrak{C}}\psi(\vec{n})\ket{\vec{n}},
\end{equation}
and then the reduced state of the system will be
\begin{equation}
\rho_{\mathcal{S}}(\psi) = \operatorname{Tr}_{\mathcal{E}}(\ket{\psi}\bra{\psi}) = \sum_{\vec{n}\in \mathfrak{C}} |\psi(\vec{n})|^2  \operatorname{Tr}_{\mathcal{E}}\ket{\vec{n}}\bra{\vec{n}},
\end{equation}
which means that the reduced state is immediately stationary. Therefore if we are able to find the largest minimum distance code, we will be alternatively answering what is the largest size of the Hilbert subspace spanned by the codewords $\{x^{(i)}\}$. Fortunately, previous works \cite{barg_random_2002} provide the estimate size of the largest random minimum code that can be built 
in a memoryless binary symmetric channel, calculations that can be adapted to our purpose and that will let us show that is possible to build an exponentially large Hilbert subspace in which ultra-orthogonality holds exactly.\\
 %   What this tells us is that the case of $\operatorname{Tr}_{\mathcal{E}}\ket{E_k}\bra{E_\ell} = 0$ can be treated as if these cross terms were errors in a minimum distance code. Since we are interested in the union of all the Hilbert spaces such that ultra-orthogonality holds, we can then ask ourself about the biggest code with minimum distance $L$. 
    %Previous works \cite{barg_random_2002} compute Minimum distances, distance distribution and error exponents on a binary-symmetric channel for typical code from Shannon's random code ensemble and for typical codes from a random linear code ensemble, calculations that can be easily modified to our purpose and that will let us prove that indeed the size of the Hilbert space in which super-orthogonality holds is exponentially large.\\






%In order provide an insight about what can be done to tackle these questions, we are going to use some well known results in coding theory to relate Fermionic states with binary codes. Fermions are fundamental particles which can be used to represent quantum information, and thus have been used for studies in modern communication purposes\cite{knill_fermionic_2001, terhal_classical_2002, knill_efficient_2000, knill_fermionic_2001-1, knill_scheme_2001-2} . Fermionic systems are usually tackled by a description on the Fock Space, where creation and annihilation operators $\hat{a}_j ^\dagger$ and $\hat{a}_j$, satisfying Fermi-Dirac commutation relations ($\{\hat{a}_j,\hat{a}_k\}=0$ and $\{\hat{a}_j,\hat{a}^{\dagger}_k\}=\delta_{jk}$), are used to describe observables such as the Occupation number ($\hat{N}_j=\hat{a}_{j}^{\dagger}\hat{a}_j$)\cite{fradkin_field_1997}. Due to the Fermi-Dirac commutation relation and the Pauli exclusion principle, the algebra generated by $\hat{a}_1,\hat{a}_2,\ldots\hat{a}_N,\mathbb{I}$ fulfil the condition that $N_j=\{0,1\}$\cite{reyes-lega_aspects_2016}, then broadly speaking, we say that when we work with Fermionic systems is like dealing with binary sequences.\\




%Particularly in the case of quadratic Hamiltonians, it is convenient to introduce Majorana operators which are related to the creation and annihilation operators as
%\begin{equation}
%\hat{c}_{2j-1}=\hat{a}_j^{\dagger}+\hat{a}_j\quad\quad \hat{c}_{2j}=(-i)(\hat{a}_j^{\dagger} -\hat{a}_j).
%\end{equation}
%\indent They can be interpreted as an analogous to to coordinate and momentum operators for bosonic modes. The algebra generated by these operators is known as the Clifford algebra ($\mathcal{C}_{2N}$)\cite{bravyi_classical_2005} and it correspondent Fermi-Dirac commutation relation is given by
%\begin{equation}
%\hat{c}_a\hat{c}_b + \hat{c}_b\hat{c}_a =2\delta_{ab}.
%\end{equation}
%\indent As a consequence, an arbitrary operator $X \in \mathcal{C}_{2N}$ can be represented as a polynomial in $\{\hat{c}_ a\}$\cite{bravyi_classical_2005}, namely
%\begin{equation}
%X=\alpha \hat{I} + \sum_{1\leq a_1\leq\ldots \leq a_p \leq 2N} \alpha_{a_1,\ldots a_p}\hat{c}_{a_1}  \ldots  \hat{c}_{a_p}.
%\label{Intro:Grassmann}
%\end{equation}
%\indent If we look back to our problem, we see that particularly, the term $\operatorname{Tr}_{\mathcal{E}} \ket{E_i}\bra{E_j}$ could be expanded in terms of some operators $\gamma$ as in \eqref{Intro:Grassmann}. 


%  Namely, in terms of our problem, we would like to explore the behaviour of the terms , and to do so, we could expand 

% Broadly speaking what this tell us is that when we work over fermion systems, we are always dealing with large  sequences of zeros and ones, meaning that in some way results used in binary code theory could be used for the case of Fermionic systems
\indent This document is divided in four chapters: In the first chapter, we will introduce canonical typicality and its relation to statistical mechanics. Starting with a discussion of the thermodynamic entropy and the main differences between Boltzmann's  and Gibb's approach to thermodynamics. Next we introduce typicality and its foundations, and finally a detailed explanation of the problem of ultra-orthogonality. In the second chapter, we will provide all the theoretical background concerning fermionic systems and code theory, in particular, we will be discussing the case of the one dimensional $XY$ model, and we will finish with the introduction of some important results in code theory. In the third chapter, we present our main results which are an explicit connection between a special kind of fermionic systems and random minimum distance codes. Also, we will compute an estimate size of the Hilbert subspace associated with the states that fulfil ultra-orthogonality. Finally, the fourth chapter contains the conclusions and further perspectives of our work.

% more over, in this chapter we present our main result which we exhibit as follows, first, an explicit connection between the Fermionic systems and random minimum distance codes is formulated in order to understand ultra-orthogonality for Fermionic systems, and second, we provide an estimation of the size of the Hilbert subspace associated with the states that fulfil ultra-orthogonality. Finally, in the fourth chapter we present our conclusions as well as further perspectives of our work.


