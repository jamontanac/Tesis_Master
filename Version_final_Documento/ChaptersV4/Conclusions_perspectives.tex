\chapter{Conclusions and Perspectives}

Throughout all this document, we have presented a series of arguments to show, that ultra-orthogonality is intimately related with the mechanism of equilibration as an instantaneous phenomenon.  As we mentioned, ultra-orthogonality is connected to thermalisation problem, since the overwhelming  majority of states in Hilbert space, are such that its correspondent reduced states approximately coincide with the thermal state, hence, the state of equilibrium we expect to have for most cases should be the canonical state. Particularly, we addressed the problem when Ultra Orthogonality holds exactly, that is, the equilibrium state is automatically reached when computing the correspondent reduced state of the Universe.\\
\indent This work was  devoted to show that for the a special kind of Fermionic systems, there exists a particular sort of pure dynamical states, such that when we take the correspondent partial trace over its environment, the resultant reduced states are automatically in its correspondent state of equilibrium when the condition of having more differences than the number of modes in the subsystem of $L$ modes is fulfilled. In our results an interpretation in terms of binary sequences of the excited states was needed to make our conclusions.\\
\indent Not satisfied with finding this result we decided to estimate the size of the correspondent Hilbert subspace associated with the states that fulfil exactly the property of ultra-orthogonality, in the sense that all reduced states, will be automatically stationary states (constant states over time). By using the formalism of random-coding exponents, explained in chapter two, it was possible to provide an expression to the large deviation present in the Fermionic systems. Namely, the exponent provided an estimate of the expected number of random Fermionic minimum codes that fulfil ultra-orthogonality. Particularly we showed that whenever this exponent is larger than zero, it is possible to show that there is exponentially large Hilbert subspaces fulfilling the condition of ultra-orthogonality. In other words, the main result of our work is that it is possible to build exponentially large Hilbert subspaces in which any fully interacting state will be immediately stationary when we compute the partial trace over the correspondent $N-L$ modes.\\

\indent The result in this work brought more questions than answers to us, where one of the first questions was ``what happens if think this problem in terms of linear codes in which a restriction over the energy is imposed?''. This particular question came to our mind because as it is shown in the work of A. Brag. et. al. \cite{barg_random_2002} linear random codes have a better exponent of error than the one obtained from the random codes. From our point of view, we consider that this could guide to interesting insights about equilibration in these kind of systems. However, this particular problem was not addressed since liner superposition of codewords may end up not conserving the restriction of energy, therefore we consider that another study will be needed to confirm if this can be done and provide the respective insights to equilibration.\\
\indent We want to point out that the physical meaning of these quantities was not fully understood and the question ``what exactly is the physical meaning of these exponents?'' remained open for us. What we want to stress here is that indeed, there was an attempt to find some physical interpretation about these quantities but not conclusive result was found.\\

\indent Finally, the case when the number of differences between two random fermionic codewords is less than $L$ remain an open problem. Nonetheless, we strongly belief that this case has to lead to similar conclusion about ultra-orthogonality, that is, we expect that when the number of differences $d$ is less than the number of modes in the subsystem $L$, ultra-orthogonality has also to appear.




